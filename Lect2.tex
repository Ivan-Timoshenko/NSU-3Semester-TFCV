\section{Множества на $\overline{\mathbb{C}}$}

\begin{definition}
    \begin{itemize}
        \item $\delta$ - окрестность точки $z_0 \in \mathbb{C}$ - это множество $C(\delta, z_0) = \{z \in \mathbb{C} \ | \ \left| z - z_0\right| < \delta\}, \delta > 0$.
        \item Проколотая окрестность точки $z_0 \in \mathbb{C}$ - это множество $C^*(\delta, z_0) = C(\delta, z_0) \backslash \{z_0\}$
        \item Окрестность точки \{$\infty$\} - множество $\{z \in \mathbb{C} \ | \ \left| z \right| > \delta\}$
        \item Точка $z$ - изолированная точка множества $E \subset \overline{\mathbb{C}}$, если $\exists \delta > 0: C(\delta, z_0) \cap E = \{z_0\}$
    \end{itemize}
\end{definition}

\begin{definition}
    Точка $z$ называется предельной точкой множества $E \subset \overline{\mathbb{C}}$, если $\forall \delta > 0 \ C^*(\delta, z) \cap E \neq \varnothing$
\end{definition}

\begin{definition}
    Точка $z$ называется внутренней точкой множества $E \subset \overline{\mathbb{C}}$, если $\exists \delta > 0 \ C(\delta, z) \subset E$
\end{definition}

\begin{definition}
    Точка $z$ называется внешней точкой множества $E \subset \overline{\mathbb{C}}$, если $\exists \delta > 0 \ C(\delta, z) \cap E = \varnothing$
\end{definition}

\begin{definition}
    Точка $z$ называется граничной точкой $E$, если 
    \[\forall \delta > 0 \begin{cases}
        C(\delta, z) \cap E \neq \varnothing \\
        C(\delta, z) \backslash E \neq \varnothing
    \end{cases}\]
\end{definition}

\begin{note}
    Если граничная точка $z \notin E$, то она является предельной для $E$.
\end{note}

\begin{definition}
    Граница $E$ - это совокупность всех граничных точек, обозначается $\partial E$.
\end{definition}

\begin{definition}
    Множестов называется ограниченным, если $\exists M \ (0 < M < \infty)$ - число, такое, что\\
    $\forall z \in E \ \left| z \right| < M$
\end{definition}

\begin{definition}
    Множество $E$ называется замкнутым, если оно содержит все свои граничные точки (или их нет).
\end{definition}

\begin{definition}
    Множество $E$ называется открытым, если $\forall z \in E$ $z$ является внутренней точкой для $E$, то есть 
    $\forall z \in E \ \exists \delta > 0: \ C(\delta, z) \subset E$
\end{definition}

\begin{definition}
    Замыкание множества $E$ это $\overline{E} = E \cup \partial E$.
\end{definition}

\begin{definition}
    Диаметр множества $d(E) = \underset{z, \xi \in E}{\sup} \left| z - \xi\right|$
\end{definition}

\begin{definition}
    Расстоянием между множествами $E$ и $G$ называется $D(E, G) = \underset{z \in E, \zeta \in G}{\inf} \left|z - \zeta\right|$
\end{definition}

\begin{definition}
    Множество $E$ называется связным, если его нельзя представить как $E = A \cup B, \ A, B \subset E$, таких, что 
    \begin{enumerate}
        \item $A, B \neq \varnothing$
        \item $A \cap B = \varnothing$
        \item $A$ и $B$ не содержат предельных точек друг друга.
    \end{enumerate}  
\end{definition}

\begin{definition}
    Множество $E$ называется линейно связным, если $\forall z_1, z_2 \in E \ \exists$ непрерывная функция $\varphi: [0, 1] \to E$ такая, что $\varphi(0) = z_1, \ \varphi(1) = z_2$
\end{definition}

\begin{definition}
    Областью называется открытое связное множество.
\end{definition}

\begin{definition}
    Компонента множества $E$ - $\forall$ максимальное по включению связное подмножество $E$.\\
    Область $E \neq \overline{\mathbb{C}}$ $n$-связная, если граница $\partial E$ состоит из $n$ компонент ($\overline{\mathbb{C}}$ считаем 1-связным).
\end{definition}

\begin{statement}
    $\forall$ множества $E \subset \overline{\mathbb{C}}$ граница $\partial E$ является замкнутым множеством.
    \begin{proof}
        Доказывается от противного: \\
        Допустим, $\exists$ предельная точка $z_0$ для $\partial E$: $z_0 \notin \partial E$.
    \end{proof}
\end{statement}

\begin{theorem}[Принцип Больцано-Вейерштрасса]
    У любого бесконечного множества $E \subset \overline{\mathbb{C}} \ \exists$ хотя бы одна предельная точка.
\end{theorem}

\begin{theorem}[Лемма Гейне-Бореля]
    Из бесконечного открытого покрытия замкнутого множества точек на $\overline{\mathbb{C}}$ можно выделить конечное подпокрытие.

    \textbf{Следствие:} На $\overline{\mathbb{C}}$ любое замкнутое и ограниченное множество является компактом.
\end{theorem}

\section{Предельные ряды комплексных чисел}

\begin{definition}
    Последовательность $\{z_n\}$ комплексных чисел $z_n = x_n + i y_n, \ n \in \mathbb{N}$ называется сходящейся к пределу $\alpha = a + ib$, если
    \[\forall \varepsilon > 0 \exists N: \left| z_n - \alpha \right| < \varepsilon \ \forall n > N\]
    Обозначается $\lim_{n \to \infty} z_n = \alpha$

    
\end{definition}
    
\textbf{Следствие:} $\lim_{n \to \infty} z_n = 0 \iff \lim_{n \to \infty} \left| z_n \right| = 0$
        
    
Попробуем перенести теорию последовательностей вещественных числе на комплексные числа:
\[\exists \lim_{n \to \infty} z_n = \alpha \implies \begin{cases}
    \exists \lim_{n \to \infty} x_n = a \\
    \exists \lim_{n \to \infty} y_n = b
\end{cases} \implies \] 
\[ \implies \begin{cases}
    \forall \varepsilon > 0 \ \exists N_1 \in \mathbb{N} \ \left| x_n - a \right| < \frac{\varepsilon}{2} \ \forall n > N_1 \\
    \forall \varepsilon > 0 \ \exists N_2 \in \mathbb{N} \ \left| y_n - b \right| < \frac{\varepsilon}{2} \ \forall n > N_2 
\end{cases} \implies \left| z_n - \alpha\right| \leq \left| x_n - a \right| + \left| y_n - b\right| < \varepsilon\]
Значит $\forall \varepsilon > 0 \ \exists N = \max(N_1, N_2) \ \left|z_n - \alpha\right| < \varepsilon \ \forall n > N$

Таким образом $\begin{cases}
    \exists \lim_{n \to \infty} x_n = a \\
    \exists \lim_{n \to \infty} y_n = b
\end{cases}  \implies \exists \lim_{n \to \infty} z_n = \alpha$
Это позволяет свести теорию последовательностей $\{z_n\}$ комплексных чисел к $\mathbb{R}$, 
то есть все теоремы о сходимости вещественных последовательностей переносятся на комплексные числа.

\begin{theorem*}[Критерий Коши]
    \[\text{Последовательность } \{z_n\} \text{ сходится} \iff \ \forall\varepsilon > 0 \ \exists N \in \mathbb{N}: \left| z_{n+m} - z_n \right|< \varepsilon  \text{ выполнено } \forall n, m > N\].
\end{theorem*}

\begin{definition}
    Последовательность $\{z_n\}$ сходится к $\infty$ ($\lim_{n \to \infty} z_n = \infty$) если
    \[\forall R > 0 \ \exists N \ \left| z_n \right| > R \text{ выполнено для } \forall n > N\]
    \[\text{При } z_n \neq 0 \ \lim_{n \to \infty}z_n = \infty \iff \lim_{n \to \infty} \frac{1}{z_n} = 0\]
\end{definition}

\newpage
\subsection{Ряды комплексных чисел}
Если у знака суммы ряда не указаны границы - считать, что слагаемые суммируются по индексу $k$, $k$ пробегает набор от 1 (иногда 0) до $\infty$
\begin{definition}
    Ряд комплексных чисел $\sum_{k=1}^{\infty}\alpha_k$ сходится (расходится), если сходится (расходится) последовательность частичных сумм $S_n = \sum_{k=1}^{n}\alpha_k$, т.е. $\exists \lim_{n \to \infty} S_n = S$.
\end{definition}

\begin{theorem*}[Критерий Коши сходимости ряда]
    \[\sum_{k=1}^{\infty} \text{ сходится } \iff \forall \varepsilon>0 \ \exists N \in \mathbb{N} \begin{cases*}
        \forall n > N \\
        \forall k \in \mathbb{N}
    \end{cases*} \left| S_{n+k} - S_n\right| < \varepsilon\]
\end{theorem*}

\begin{definition}
    Ряд $\alpha_k$ сходится абсолютно, если сходится ряд $\sum_{k = 1}^{\infty} \left| \alpha_k\right|$. Но поскольку $\left| \sum_{k = 1}^{m} \alpha_{n+k}\right| \leq \sum_{k=1}^{m}\left| \alpha_{n+k} \right|$, то из абсолютной сходимости ряда следует общая сходимость.
\end{definition}

\begin{statement}
    Член ряда $\alpha_k$ представим в виде $a_k + ib_k$. Тогда:
    \[\begin{cases*}
        \left| a_k \right| \leq \left|\alpha_k\right| \leq \left| a_k \right| + \left| b_k \right| \\
        \left| b_k \right| \leq \left|\alpha_k\right| \leq \left| a_k \right| + \left| b_k \right|
    \end{cases*}\]
    Значит, если $\sum_{k=1}^{\infty}\alpha_k$ абсолютно сходится, то $\sum_{k=1}^{\infty}a_k$ и  $\sum_{k=1}^{\infty}b_k$ тоже абсолютно сходятся. 
    В свою очередь, из абсолютной сходимости $\sum_{k=1}^{\infty}a_k$ и  $\sum_{k=1}^{\infty}b_k$ следует абсолютная сходимость $\sum_{k=1}^{\infty}\alpha_k$.
\end{statement}

\begin{definition}
    Произведение двух сходящихся рядов сходится к произведению их сумм.
\end{definition}

Стандартные разложения функций от комплексных чисел:
\begin{enumerate}
    \item $e^\alpha = 1 + \alpha + \frac{\alpha^2}{2!} + \hdots + \frac{\alpha^n}{n!} + \hdots = \sum_{k=0}^{\infty}\frac{\alpha^k}{k!}$
    \item $\sin(\alpha) = \alpha - \frac{\alpha^3}{3!} + \frac{\alpha^5}{5!} + \hdots = \sum_{k=0}^{\infty}\frac{(-1)^k\cdot\alpha^{2k+1}}{(2k+1)!}$
    \item $\cos(\alpha) = \sum_{k=0}^{\infty}\frac{(-1)^k\cdot\alpha^{2k}}{2k!}$
    \item $e^{i\alpha} = \sum\frac{(i\alpha)^k}{k!} = \sum\frac{(-1)^k\cdot\alpha^{2k}}{2k!} + i\sum\frac{(-1)^k\cdot\alpha^{(2k+1)}}{(2k+1)!} = \cos(\alpha) + i\sin(\alpha)$
\end{enumerate}

Пусть $\mid \alpha \mid = R, \ \arg(\alpha) = \varphi$. Тогда:
\[\alpha = R\cdot e^{i\cdot\arg(\alpha)} = R(\cos(\varphi) + i\sin(\varphi))\]
И поскольку можем представить $\alpha = a + ib$:
\[e^{\alpha} = e^a\cdot (\cos(b) + i\sin(b)), \quad \left| e^\alpha\right| = e^a, \ \arg(e^\alpha) = b\]
Поэтому $e^{2\pi k\cdot i} = 1$