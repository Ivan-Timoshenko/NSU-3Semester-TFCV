\textbf{Правило дифференциирования монома:\\}
\bigskip
Пусть $f(x)  = x_1^{i_1} \cdot x_2^{i_2} \cdot \dots \cdot x_m^{i_m}, \ x = (x_1, \dots, x_m)$. Тогда 
\[\frac{\partial^{i_1+i_2+ \dots + i_m} f}{\partial x_1^{i_1} \dots \partial x_m^{i_m}}(0) = i_1! \cdot \hdots \cdot i_m!\]
Любая другая производная любого порядка в точке $0$ равна $0$.

\subsection{Мульти-индексы}
Придумаем $\mu = (i_1, \hdots, i_m)$ - численный вектор, в котором $\forall j = \overline{1 \hdots m} \ i_j \geq 0$ и назовем его \textbf{мультииндексом}.
\par
Для мультииндексов определены операции:
\[\mu! := i_1! \cdot \hdots \cdot  i_m! \qquad \left| \mu \right| = \sum_{j = 1}^{m} i_j \text{ - порядок мультииндекса}\]
\[x \in \mathbb{R}^m, x = (x_1, \hdots, x_m), \quad x^{\mu} = x_1^{i_1} \cdot \hdots \cdot x_m^{i_m}\]
\[C^{\mu}_k = \frac{k!}{\mu!} = \frac{k!}{i_1!\cdot \hdots \cdot i_m!} \text{, где } k = \left|\mu\right|\]

\bigskip

Зададим контекст:
\[f:\mathbb{R}^m \to E, \ f \in D^k(p), \ \mu = (i_1, \hdots, i_m)\]
Тогда:
\[\frac{\partial^{\mu}f}{\partial x^{\mu}} = D^{\mu}f = \frac{\partial^{i_1}}{\partial x_1^{i_1}}(\frac{\partial^{i_2}}{\partial x_2^{i_2}}( \hdots (\frac{\partial^{i_m}}{\partial x_m^{i_m}}f)))\]

\begin{theorem*}[разложение Тейлора]
    $\exists!$ многочлен $A(x)$ степени $\leq k$ такой, что $f(x) - A(x) \underset{x \to p}{=} o(x-p)^k$    
    \[A(x) = f(p) + f'(p)(x-p) + \frac{f''(p)}{2!}(x-p)^2 + \hdots + \frac{f^{(k)}(p)}{k!}(x-p)^k\]
\end{theorem*}

\begin{theorem*}[разложение Тейлора для нескольких переменных]
    Пусть $f: \mathbb{R}^m \to E, \ f \in D^k(p)$, тогда $\exists!$ многочлен $A(x) \ deg(A) \leq k$,
    такой, что $f(x) - A(x) \underset{x \to p}{=} o(\left| x - p\right|^k)$:
    \[A(x) = f(p) + \frac{df(p) \langle x-p \rangle}{1!} + \frac{d^f(p)\langle x-p \rangle}{2!} + \hdots + \frac{d^kf(p)\langle x-p \rangle}{k!}\]
    \begin{proof}
        \textbf{Единственность:} Пусть есть два таких многочлена $A(x), B(x)$. Введем $C(x) := A(x) - B(x) = o(\left| x - p\right|^k)$. И докажем вспомогательное утверждение:
        \begin{statement*}
            $degC \leq k \ C(x) \underset{x \to p}{=} o(\left|x-p\right|^k)$, тогда $C \equiv 0$.
            \begin{proof}
                \begin{enumerate}
                    \item Фиксируем $v \in \mathbb{R}^m$ и рассмотрим $h(t) = C(p+tv)$ - многочлен одной переменной.
                    По условию $h(t) = o(t^k)$ для одной переменной (доказывали это в первом семестре), т.е. $h(t) \equiv 0$. В частности, при $t = 1 \ h(t) = C(p+v) = 0$.
                    \item Поскольку 1. выполняется $\forall v$, то $C(p+v) = 0 \  \forall v$.
                \end{enumerate}
            \end{proof}
            Тогда в силу доказанного утверждения получаем единственность.
        \end{statement*}
        \par
        \textbf{Существование:}
        Введем $g(x) = f(x) - A(x), \ f:\mathbb{R}^m \to E$. $g(p) = 0$ и все производные до порядка $k$
        включительно равны $0$ в $p$, $g \in D^K(p)$. Необходимо доказать, что из этого следует, что $g(x) = o(\left| x - p\right|^k)$.
        \\
        Пусть $\varepsilon > 0$, надо показать, что $\left| g(x) \right| < \varepsilon\cdot \left| x - p\right|^k$ в некоторой $U$ - окрестности точки $p$.
        Пусть $\varepsilon_{k-1}(x)$ - какая-то производная порядка $k-1$ функции $g$, $\varepsilon_{k-1}$ определена в некотором шаре $V_p$ с центром в $p$.
        \begin{equation}
            \label{eq1}
            \varepsilon_{k-1}(p) = 0, \ \frac{\partial \varepsilon_{k-1}}{\partial x_i}(p) = 0 \ \forall i = 1 \hdots m
        \end{equation}
        Поэтому имеется маленький шар $U \subset V_p$ в котором выполнено:
        \begin{equation}
            \label{eq2}
            \forall x \in U \ \left| \varepsilon_{k-1}\right|(x) \leq \varepsilon\cdot \left| x - p \right|
        \end{equation}
        В самом деле, $\varepsilon_{k-1}(x) = \varepsilon_{k-1}(p) + d\varepsilon_{k-1}(p)\langle x - p \rangle + o(\left| x - p \right|)$, причем первое слагаемое равно нулю из того, что "все производные до порядка $k$ включительно равны $0$ в $p$", а второе - из уравнения (\ref{eq}).
        Значит $\varepsilon_{k-1}(x) = o(\left| x - p\right|) \implies \varepsilon_{k-1}(x) \leq \varepsilon\left| x - p \right|$
        Итак, ясно, что существует шарик $U$, в котором все производные $k-1$ порядка 
        имеют оценку (\ref{eq2})
        Пусть $\varepsilon_{k-2}$ - какая-то производная функции $g$ порядка $k-2$. Все ее первые частные производные по доказанному в шаре $U$ оцениваются в $\varepsilon\cdot \left| x - p\right|$.
        По лемме о степенной оценке приращения для $\varepsilon_{k-2}$ выполнено в шаре $U$:
        \[\left| \varepsilon_{k-2}(x)\right| \leq \left| \frac{\varepsilon \left| x - p\right|^2}{2}\right|\]
        Для $k-3, k-4, \hdots$ аналогично. 
        \[\left| g(x) \right| = \left| g^{(k-k)}(x) \right| \leq \varepsilon \cdot \frac{\left| x- p \right| ^ k}{k!} \leq \varepsilon\cdot \left| x - p\right|^k
        \]
    \end{proof}
\end{theorem*}

\begin{theorem}[Достаточное условие локального экстремума функции многих переменных]    
    Пусть $f \in D^2(p), \ f:\mathbb{R}^m \to R$ и $df(p) > 0$. Тогда:
    \begin{enumerate}
        \item $d^2f(p) > 0$ - строгий локальный минимум
        \item $d^2f(p) < 0$ - строгий локальный максимум
        \item Если $d^2f(p)$ знаконеопределен, т.е. $\exists u \in \mathbb{R}^m \ d^2f(p)\langle u \rangle > 0$ и $\exists v \in \mathbb{R}^m \ d^2f(p)\langle v \rangle < 0$, то $p$ - седловая точка.
    \end{enumerate}
    \begin{proof}
        
        Докажем пункт 3:
        \begin{proof}
            Пусть $df(p) = 0$ и существуют вектора $u$ и $v$, такие, что $d^2f(p)\langle u \rangle > 0, \ d^2f(p)\langle v \rangle < 0$.
            
            Введем функцию $h(t) = f(p + tu)$. Тогда $h'(0) = df(p)\langle u \rangle = 0, \  h''(0) = d^2f(p)\langle u \rangle > 0$. Значит у функции $h$ в точке $0$ строгий минимум 
            (по достаточному условию экстремума для одной переменной). Аналогично вдоль $p + tv$ функция имеет строгий максимум, значит $p$ - седловая точка.
        \end{proof}
        
        Докажем пункт 1:
        \begin{proof}
            Пусть $d^2f(p) > 0$, то есть $\forall v \neq 0 \ d^f(p)\langle v \rangle > 0$. Сфера $S^{m-1} = \{v \in \mathbb{R}^m \quad \left| v \right| = 1\}$ - компактна (замкнута и ограничена). $d^2f(p):S^{m-1} \to \mathbb{R}$ - однородный многочлен второго порядка.
            Так как $d^2f$ - непрерывная функция на компакте, то у нее $\exists \min = C > 0$, т.е.\\ $\forall v \in S^{m-1} \quad d^2f(p)\langle v \rangle \geq C$.
            \begin{statement*}
                Тогда $\forall v \neq 0 \ d^2f(p)\langle v \rangle \geq C\cdot \left| v \right|^2$
                \begin{proof}
                    \[\forall v \neq 0 \quad d^2f(p)\langle v \rangle = d^f(p)\langle \left| v \right| \cdot \frac{v}{\left| v \right|} \rangle = 
                    \left| v \right|^2\cdot d^2f(p)\langle \frac{v}{\left| v \right|} \rangle \geq C\cdot \left| v \right|^2\]
                \end{proof}
            \end{statement*}

            Значит 
            \[
                f(x) = f(p) + df(p)\langle x -p \rangle + \frac{d^2f(p)\langle x - p \rangle}{2!} + \alpha(x)\left| x - p\right|^2, \ \alpha(x) \underset{x \to p }{\to} O(1)
            \]
            \[f(x) \geq f(p) + 0 + \frac{C}{2!}\cdot \left| x - p \right|^2 + \alpha(x)\left| x - p\right|^2\]
            Существует окрестность $U$ точки $p$, такая, что $\left| \alpha(x) \right| \leq \frac{C}{3} \ \forall x \in U$. Тогда для $\forall x \in U$:
            \[f(x) \geq f(p) + \frac{C}{2!}\left| x - p \right|^2 - \frac{C}{3}\left| x - p \right|^2 = f(p) + \frac{C}{6}\left| x - p\right| ^2\]
            То есть $f(x) - f(p) \geq \frac{C}{6}\left| x - p \right|^2 > 0 \implies$ в $U \ f(p) < f(x) \ \forall x \in U$. Пункт 1 доказан. 
        \end{proof}
        Пункт 2 доказывается аналогично пункту 1.
    \end{proof}
\end{theorem}

\begin{theorem}[Полиномиальное разложение композиции]
    Пусть $k \geq 0, f, g$ - функции, $A(x), B(y)$ - полиномы. Предположим, что $f$ и $A$ в точке $p$ имеют порядок касания $\geq k$, $g$ и $B$ в точке $p$ имеют порядок касания $\geq k$. То есть:
    \[f(x) - A(x) = \alpha(x) \underset{x \to p}{=} o(\left| x - p\right|^k), \ \alpha(p) = 0\] 
    \[g(y) - B(y) = \beta(y) \underset{y \to q}{=} o(\left| y - q \right|^k), \ \beta(q) = 0\]
    Тогда $g\circ f$ имеет с $B \circ A$ порядок касания $\geq k$. 
    \begin{proof}
        При $k = 0$:
        \[\alpha(x) = o(1) \implies f(x) - A(x) \underset{x \to p}{=} 0 \implies f(x) \underset{x \to p}{\to} f(p)\]
        Для функции $g$ аналогично, после чего применяем теорему о непрерывности композиции. Для $ k = 0$ доказано.
\newline
        Пусть $k \geq 0$. Тогда:
        \[\begin{cases}
            f(x) = \alpha(x) + A(x) \\
            g(y) = \beta(x) + B(x)
        \end{cases} \implies
        \begin{cases}
            \alpha(x) = o(\left| x - p \right|) \\
            \beta(x) = o(\left| y - q \right|)
        \end{cases} \implies 
        \begin{cases}
            \alpha \in D^1(p) \\
            \beta \in D^1(q)
        \end{cases} \implies 
        \begin{cases}
            f \in D^1(p) \\
            g \in D^1(q)
        \end{cases}\]
        \[g(f(x)) - B(A(x)) = 
        g(f(x)) - B(f(x)) + B(f(x)) - B(A(x)) = 
        \beta(f(x)) +B(f(x)) + B(A(x))\]
        Заметим, что $\beta(f(x)) = o(\left| f(x) - q\right|^k) \underset{x \to p }{=} o(\left| f(x)  - f(p) \right|^k)$. 
        При этом $f(x) - f(p) = O(\left| x - p \right|^1)$, так как $f \in D^1(p)$. А значит:
        \[\beta(f(x)) = o(O(\left| x - p \right|^k)) = o(\left| x - p \right|^k)\]
        \newline
        Пусть теперь $V$ - шар конечного радиуса с центром в $q$. Все частные производные многочлена 
        $B$ в шаре $V$ ограничены некоторой константой $C$. Тогда по лемме об оценке приращения:
        \[\forall y_1, y_2 \in V \quad \left| B(y_1) - B(y_2) \right| = O(\left| y_1 - y_2 \right|)\]
        При $x \to p \begin{cases}
            f(x) \to q \\
            A(x) \to q
        \end{cases}$ и поэтому $f(x) , B(x) \in V$.
        В таком случае \[
        B(f(x)) - B(A(x)) \underset{x \to p}{=} O(f(x) - A(x) ) = O(o(\left| x - p \right|^k))\]
    \end{proof}
\end{theorem}