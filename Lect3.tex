\section{Функции комплексного переменного}
Пусть $E \subset \mathbb{C}$ и $f: E \to \mathbb{C}$ (т.е. $z \to f(z)$).
Если $f$ - инъекция, т.е. $z_1 \neq z_2 \implies f(z_1) \neq f(z_2)$, то говорят, что $f$ - однослойна.
Если $f$ действует из $E$ в $f(E)$ (то есть еще и сюръективна, а значит - биекция), то $\exists f^{-1}: f(E) \to E$.

\begin{definition}
    Если $z_0$ - предельная точка множества $E$ и $\lim_{z \to z_0}f(x) = f(z_0)$, то $f(z)$ непрерывна в $z_0$.
    \[\lim_{z \to z_0}f(z) = f(z_0) \iff \forall \varepsilon > 0 \ \exists \delta: \left| z - z_0 \right| < \delta \implies \left| f(z) - f(z_0)\right| < \varepsilon\]
    Функция $f(z)$ представима в виде $u(z) + iv(z)$ (мнимая и комплексная части), а значит $f$ непрерывна на $E \iff u(z), v(z), \left| f(z) \right| $ - непрерывны на $E$.
\end{definition}

Если $E$ - компактно, а $f$ - непрерывно, то:
\begin{enumerate}
    \item $\left| f(z) \right|$ достигает $\max$ и $\min$ на $E$
    \item $f(z)$ равномерно непрерывна на $E$ по т. Кантора
\end{enumerate}

\begin{definition}
Функция $\gamma(t) = x(t) + iy(t), \ t\in [\alpha, \beta]$ - задает кривую на комплексной плоскости, если $x(t), y(t)$ - непрерывны.
    
\begin{itemize}
    \item Если $\gamma(t)$ инъективна (за исключением $\gamma(\alpha) = \gamma(\beta)$), то $\gamma$ - жорданова кривая.
    \item $\gamma(t)$ называется гладкой, если $x(t), y(t)$ непрерывно дифферециируемы и $\forall t \ \gamma'(t) = x'(t) + iy'(t) \neq 0$
\end{itemize}
\end{definition}

\begin{theorem*}[о стандартном радиусе]
    Пусть $\Gamma(t), t \in [\alpha, \beta]$ - гладкая замкнутая жорданова кривая. Тогда
    \[\forall \theta_0 \in (0, \frac{\pi}{2}) \ \exists \delta_0 \text{ такаая, что}\]
    \begin{enumerate}
        \item Окружность $\left| z - z_0 \right| = \underset{z_0 \in \Gamma}{R} \leq \delta_0$ пересекает $\Gamma$ строго в двух точках
        \item При переходе от $z_1 \in \Gamma$ к $z_2 \in \Gamma$, где $\left| z_2 - z_1 \right| < \delta_0$ угол наклона касательной меняется меньше, чем $\theta_0$
        \item Если $\left| z - z_0 \right| < \delta_0$ то $dS(s) < \frac{dr}{d\cos(\theta_0)}$
    \end{enumerate}
\end{theorem*}

\subsection{Функциональные ряды}

\begin{definition}
    Ряд $\sum f_n(z)$ сходится на $E \subset \mathbb{C}$, если он сходится $\forall z \in E$. Сходится равномерно на $E$, если:
    \[\forall \varepsilon > 0 \ \exists N \mid \forall n > N \ \forall z \ \left|S_n(z) - S(z)\right| < \varepsilon\]
    Или ряд $\sum f_n(z)$ сходится равномерно, если 
    \[\forall  \varepsilon >0 \ \exists N \ \forall n > N \  \forall m \forall z \left| \sum_{k = 0}^{m} f_{n+k}(z)\right| < \varepsilon\]
\end{definition}

\begin{theorem*}[Признак равномерной сходимости Вейерштрасса]
    Если $\forall z \in E$ каждый элемент ряда, начиная с некоторого номера $n_0$ удовлетворяет неравенству
    \begin{equation}
        \label{VeierEq}
        \left| f_k(z)\right| \leq \alpha_k, \ k \geq n_0
    \end{equation}
    и числовой ряд $\sum_{k = n_0}^{\infty}\alpha_k$ сходится, то Функциональный ряд $\sum f_k(z)$ сходится равномерно и абсолютно на $E$.
\newpage
    \begin{proof}
        Так как ряд $\sum_{k=n_0}^{\infty}\alpha_k$ сходится, то по критерию Коши:
        \[\forall \varepsilon > 0 \ \exists N \in \mathbb{N} \mid \sum_{k=1}^{m}\alpha_{n+k} < \varepsilon \quad \forall n > N, \ \forall m \in \mathbb{N}\]

        \[\left| \sum_{k=1}^{m} f_{n+k}(z)\right| \leq \sum_{k=1}^{m}\left| f_{n+k}(z)\right| \overset{\text{по нер-ву \ref{VeierEq}}}{\leq} \sum_{k=1}^{m}\alpha_{n+k} < \varepsilon\]
        Следовательно, по критерию Коши $\sum f_k(z)$ сходится равномерно и абсолютно.
    \end{proof}
\end{theorem*}

\begin{theorem*}
    Сумма $S(z) = \sum f_k(z)$ равномерно сходящегося ряда непрерывных на $E$ функций $f_k(z)$ сама непрерывна на $E$.

    \begin{proof}
        Пусть $z_0 \in E$ - произвольная точка. Тогда для $z \in E$:
        \[\left| S(z) - S(z_0) \right| \leq \left|S(z) - S_N(z)\right| + \left| S_N(z) - S_N(z_0)\right| + \left|S_N(z_0) - S(z_0)\right|\]
        По определению равномерной сходимости ряда $\sum f_k(z)$:
        \[\forall \varepsilon > 0 \ \exists N \in \mathbb{N} \left| S(z) - S_N(z)\right| < \frac{\varepsilon}{3} \text{ и } \left| S_N(z_0) - S(z_0)\right| < \frac{\varepsilon}{3}\]
        Конечная сумма $S_N(z) = \sum_{k=1}^{N}f_k(z)$ непрерывных на $E$ функций $f_k(z)$ - сама непрерывна на $E$. Тогда по определению непрерывности функции:
        \[\forall \varepsilon > 0 \ \exists \delta > 0: \ \left|z - z_0\right| < \delta \implies \left|S_N(z) - S_N(z_0)\right| < \frac{\varepsilon}{3}\]
        А значит $\left|S(z) - S(z_0) \right|< \varepsilon \implies$ функция $S(z)$ непрерывна в $z_0$. В силу произвольности выбора $z_0 \in E \ S(z)$ непрерывна на $E$. 
    \end{proof}
\end{theorem*}

\begin{definition}
    Бесконечная сумма вида $\sum_{k=0}^{\infty}C_k\cdot(z-z_0)^k$ называется степенным рядом, в котором $C_k$ - коэффициенты степенного ряда. 
    \newline
    Поскольку линейная замена $t = z - z_0$ превращает этот ряд в более удобную форму $\sum C_k t^k$ - то будем рассматривать ряды именно такого вида.
\end{definition}

\begin{theorem*}[Коши-Адамара]
    Пусть дан ряд $\sum C_k z^k$ и $l = \overline{\lim}_{k \to \infty} \sqrt[k]{\left| C_k \right|}$. Тогда:
    \begin{enumerate}
        \item Если $l = 0$, то данный ряд сходится на всей комплексной плоскости ($\forall z \in \mathbb{C}$).
        \item Если $l = \infty$, то ряд сходится при $z = 0$ и расходится $\forall z \neq 0$.
        \item Если $0 < l < \infty$, то ряд абсолютно сходится в круге $\left| z \right| < \frac{1}{l}$ и расходится при $\left| z \right| > \frac{1}{l}$.
    \end{enumerate}

    Число $R = \frac{1}{\overline{\lim_{k \to \infty}} \sqrt[k]{\left| C_k \right|}}$ называется радиусом сходимости степенного ряда, а круг $\left| z - z_0 \right| < R$ называется кругом сходимости.

    \begin{proof}
        В точке $z = 0$ утверждение верно при $\forall C_k \implies \forall l$ ряд сходится, так как ряд сходится в $z = 0$. Пусть теперь $z \neq 0$.
        \par Случай 1: $l = 0$. \\
        То есть $l = \overline{\lim}_{k \to \infty} \sqrt[k]{\left| C_k \right|} = \lim_{k \to \infty} \sqrt[k]{\left| C_k \right|} = 0$. Значит:
        \[\forall z \neq \infty \ \lim_{k \to \infty} \sqrt[k]{\left| C_k z^k\right|} = \left| z \right| \cdot \lim_{k \to \infty}\sqrt[k]{\left| C_k \right|} = 0 \text{ так как $\left| z\right|  \neq \infty$, а предел равен 0}\]
        Тогда по радикальному признаку Коши для рядов с положительными членами ряд $\sum_{k=0}^{\infty}\left| C_k z^k \right|$ сходится $\forall z \neq \infty$, т.е. ряд $\sum C_k z^k$ сходится на $\mathbb{C}$.
        \newpage
        Случай 2: $l = \infty$. \par
        Если бы ряд сходился в точке $z \neq 0$, то в силу необходимого условия сходимости $\exists M > 1 \ \left| C_k z^k\right| < M$ или $\sqrt[k]{\left| C_k \right|} < \frac{M}{\left| z \right|}$, что противоречит условию $\overline{\lim}_{k \to \infty} \sqrt[k]{\left| C_k \right|} = \infty$. \par

        Случай 3: $0 < l < \infty$ \par
        Пусть $0 < \left| z \right| < \frac{1}{l}$. По определению верхнего предела:
        \[\forall \varepsilon > 0 \ \exists N \in \mathbb{N} \ \sqrt[k]{\left| C_k \right|} < \varepsilon \ \forall k > N\]
        Пусть $\varepsilon = \frac{1-  l\left| z \right|}{2 \left| z \right|} \implies \sqrt[k]{\left| C_k \right|} < l + \frac{1 - l\left| z \right|}{2 \left| z \right|} = \frac{1 + l\left| z \right|}{2 \left| z \right|}$. Тогда:
        \[\left| z \right|\sqrt[k]{\left| C_k\right|} < \frac{1 + l\left| z \right|}{2} = q < 1\]
        \[\left| C_k z^k \right| < q^k, \ k > N\]
        Значит ряд сходится абсолютно при $\left| z \right| < \frac{1}{l}$. \\
        Пусть $\left| z \right| > \frac{1}{l}$. По определению верхнего предела:
        \[ \forall \varepsilon > 0 \ \exists \text{ бесконечное множество индексов } k_n, \ n = 1 \hdots \infty \text{ таких, что } \sqrt[k_n]{\left| C_{k_n}\right|} > l - \varepsilon\]
        Пусть $\varepsilon = \frac{l\left| z \right| - 1}{\left| z \right|}$. Тогда $\sqrt[k_n]{C_{k_n}} > \frac{l\left| z \right| - l \left| z \right| + 1}{\left| z \right|} \implies \left| z \right|\sqrt[k_n]{\left| C_{k_n} \right|} > 1$. Значит ряд расходится по радикальному признаку Коши при $\left| z \right| > \frac{1}{l}$.
    \end{proof}
\end{theorem*}

\begin{theorem}[Первая теорема Абеля]
    Если степенной ряд $\sum C_k z^k$ сходится в некоторой $z_0 \in \mathbb{C}$ такой, что $z_0 \neq 0$, 
    то он сходится абсолютно в круге $\left| z \right| < \left| z_0 \right|$.

    \begin{proof}
        По теореме Коши-Адамара ряд сходится абсолютно в круге $\left| z \right| < R$ и расходится при $\left| z \right| > R$. 
        Значит $\left| z_0 \right| \leq R \implies $ ряд сходится абсолютно при $\left| z \right| < \left| z_0 \right|$.
    \end{proof}
\end{theorem}


Напомню контекст: ряд (1)$\sum C_k z^k, \ R = \frac{1}{\overline{lim}_{k \to \infty}} \sqrt[k]{\left| C_k\right|}$. 

\begin{note}
    Степенной ряд не обязательно сходится равномерно в круге $\left| z \right| < R$. 

    Определение равномерной сходимости: 
    \[\forall \varepsilon > 0 \ \exists N \left| S_n(z) - S(z) \right| < \varepsilon \ \forall n > N, \forall z \in E\]

    \begin{proof}
        Для доказательства этого замечания сначала докажем формулы геометрической прогрессии:
        \begin{statement*}
            \[S_n(q) = \sum_{k = 0}^{n}q^k = \frac{1 - q^{n+1}}{1-q}\]
            \begin{proof}
                \[S_n + q^{n+1} = 1 + q + q^2 + \hdots + q^n + q^{n+1} = 1 + q*S_n\]
                Значит $S_n = \frac{1-q^n+1}{1-q}$
            \end{proof}
        \end{statement*}
        \begin{statement*}
            \[\left|q \right| < 1 \implies S = \lim_{n \to \infty} S_n = \sum_{k=0}^{\infty} q^k = \frac{1}{1-q}\]
            \begin{proof}
                \[\lim_{n \to \infty} \sum_{k=0}^{n}q^k = \lim_{n \to \infty}\frac{1 - q^{n+1}}{1-q} = \lim_{n \to \infty}\frac{1}{1-q}(1-q^{n+1}) = \frac{1}{1-q}\lim_{n \to \infty}1-q^{n+1} = \frac{1}{1-q}\]
            \end{proof}
        \end{statement*}

        По первому утверждению: 
        \[S_n(z) = \sum_{k=0}^{n}\frac{1-z^{n+1}}{1-z} \implies \forall \in \mathbb{N} S(z) - S_n(z) = \frac{z^{n+1}}{1-z} \implies \underset{\left| z \right| < 1}{\sup}\left| S(z) - S_n(z) \right| = \underset{\left| z \right| < 1}{\sup} \frac{\left| z \right|^{n+1}}{\left| 1 - z\right|} = \infty\]
    \end{proof}
\end{note}


\begin{note}
    Степенной ряд равномерно сходится в любом замкнутом круге $\left| z \right| < r, \forall r < R$ (может быть 0).

    \begin{proof}
        Из условия теоремы ряд мажорируется рядом $\sum \left| C_k\right| r^k$, который сходится по радикальному признаку Коши. Значит степенной ряд равномерно сходится по признаку Вейерштрасса.
    \end{proof}
\end{note}

\begin{note}
    На границе круга сходимости (при $\left| z \right| = R$) степенной ряд может:
    \begin{itemize}
        \item расходится во всех точках (пример $\sum z^k)$
        \item сходится в некоторых точках границы и расходится в других (например, $\sum \frac{z^k}{k}$)
        \item сходиться во всех точках границы (например, $\sum \frac{z^k}{k^2}$)
    \end{itemize}
\end{note}


\begin{theorem}[Вторая теорема Абеля]
    Если степенной ряд с радиусом сходимости $R \in (0, \infty)$ сходится в точке $z_0$, такой, что $\left| z_0 \right| = R$, то сумма $S(z) \to S(z_0)$ при $z \to z_0$ изнутри круга по любой некасательной траектории.
\end{theorem}

Возвращаясь к формуле Эйлера:
\[e^{iz} = \cos(z) + i\sin(z)\]
Степенные ряды для $e^z, \cos(z), \sin(z)$ сходятся равномерно в $\forall $ круге $\left| z \right| < R$, где $0 < R < \infty \implies$ эти функции непрерывны на $\mathbb{C}$.

\begin{statement*}
    $e^z$ - периодичная функция с периодом $2\pi ki$
    \begin{proof}
        \[e^z\cdot e^t = e^{z+t} \implies e^{z + 2\pi ki} = e^z\cdot e^{2\pi ki} = e^z(\cos(2\pi ki) + i\sin(2\pi ki)), k \in \mathbb{Z}\]
    \end{proof}
\end{statement*}
По формуле Эйлера:
\begin{equation}
    \label{euleriz}
    e^{iz} = \cos(z) + i\sin(z)
\end{equation}
\begin{equation}
    \label{euler-iz}
    e^{-iz} = \cos(z) - i\sin(z)
\end{equation}

\begin{equation}
    \label{expcos}
    (\ref{euleriz})+(\ref{euler-iz}): e^{iz} + e^{-iz} = 2\cos(z) \implies \cos(z) = \frac{e^{iz} + e^{iz}}{2}
\end{equation}
\begin{equation}
    \label{expsin} 
    (\ref{euleriz})-(\ref{euler-iz}): e^{iz} - e^{-iz} = 2i\sin(z) \implies \sin(z) = \frac{e^{iz} - e^{-iz}}{2i}
\end{equation}
То есть формулы тригонометрии остаются в силе.

Нули функции $\sin(z)$:
Из $(\ref{expsin})$:
\[\sin(z) = 0 \implies e^{iz} - e^{-iz} = 0 \implies e^{2iz} - 1 = 0 \implies e^{2iz} = e^{2\pi ki}, k \in \mathbb{Z}\]
Значит нули функции $\sin(z)$ имеют вид $z = \pi k, \ k \in \mathbb{Z}$ Аналогично нули функции $\cos(z)$ имеют вид $z = \frac{\pi}{2} + \pi k, k \in \mathbb{Z}$

\begin{note}
    Функции $\sin(z)$ и $\cos(z)$ не ограничены на $\mathbb{C}$
    \begin{proof}
        $z$ представимо в виде $x + iy$. Из $(\ref{expcos})$:$2\cos(z) = e^{iz} + e^{-iz}$. Мы знаем, что $\left| z \right|^2 = z \cdot \overline{z}$, значит:
        \[\left| 2 \cos(z)\right|^2 = (e^{iz} + e^{-iz})\cdot(e^{i\overline{z}} + e^{-i\overline{z}}) = e^{-2y}+e^{2y} + 2\cos(2x) \underset{y \to \infty}{\to} \infty\]
    \end{proof}
\end{note}

\begin{definition}
    \[\cosh(z) = \frac{e^z + e^{-z}}{2}, \ \cosh(iz) = \cos(iz)\]
    \[\sinh(z) = \frac{e^z - e^{-z}}{2}, \ \sinh(iz) = i\sin(iz)\] 
\end{definition}