\section{Многообразия в $\mathbb{R}^n$}

\begin{definition}
    Пусть $M$ - метрическое пространство (необязательно подмножество в $\mathbb{R}^n$). $M$ является $k$-мерным мноогобразием без края,
    если $\forall p \in M$ у точки $p \ \exists $ окрестность $U_p \open M$ гомеоморфная открытому шару в $\mathbb{R}^k$.
\end{definition}

\begin{definition}
    Гомеоморфизм - непрерывное отображение, обратное к которому тоже непрерывно.
\end{definition}

\begin{theorem}[Брауэра об инвариантности области (без док-ва).]
    \par Пусть $U \open \mathbb{R}^k$ и $f:U \to \mathbb{R}^k$ - непрерывна и инъективна. \newline
    Тогда $f(U) \open \mathbb{R}^k$ и $f^{-1}: f(U) \to U$ - тоже непрерывна.
\end{theorem}

\begin{definition}
    $M$ - $k$-мерное $C^r$-мноогобразие в $R^n$, если:
    \[\forall p \in M \ \exists U \in \mathcal{N}(p), \ U \open M \text{ такая, что } U \overset{C^r}{\cong} \text{ открытому шару в } \mathbb{R}^k\]
\end{definition}

\begin{statement*}
    Предыдущее утверждение эквивалентно требованиям:
    \begin{enumerate}
        \item \[ \exists U \in \mathcal{N}(p) \ U \open M \text{ такая, что } U \cong \text{ открытому подмножеству } \omega \in \mathbb{R}^k\]
        \item \[ \exists U \in \mathcal{N}(p) \ U \open M \text{ такая, что } U \cong \mathbb{R}^k\]
    \end{enumerate}
\end{statement*}

\begin{theorem*}[1 о регулярных решениях]
    Пусть $\Omega \open \mathbb{R}^n, \ f_1, \hdots, f_k: \Omega \to \mathbb{R}$ - $C^r$ гладкие функции.
    Непустое множество $M$ задано как 
    \[M = \{\overline{x} \in \Omega \mid 
    \begin{array}{c}
        f_1(x) = 0 \\
        \vdots \\
        f_k(x) = 0
    \end{array}\}, \text{ где $\overline{x}$ - регулярная точка}\]
    То есть $\mathrm{rank} \frac{\partial f_{1 \hdots k}}{\partial x_{1 \hdots n}} = k$
    Тогда $M$ - $n-k$-мерное $C^r$-мноогобразие без края.

    \begin{proof}
        Пусть $p \in M$. Можно считать, что последние $k$ столбцов линейно независимы, то есть $\left| \frac{\partial f_{1\hdots k}}{\partial x_{n-k+1 \hdots n}}\right| \neq 0$ - определитель матрицы Якоби.  
        По теореме о неявной функции существует такая окрестность $\tilde{U}$ точки $p$, $\tilde{U} \open \mathbb{R}^n$, такая, что $\tilde{U} \cap M$ - график некоторой функции:
        \[x_{n-k+1 \hdots n} = g(x_{1 \hdots n-k})\]
        Осталось показать, что график функции является мноогобразием $U \subset \mathbb{R}^k$.
        \begin{lemma*}
            График любого $C^r$ отображения $g$, определенного на открытом подмножестве - это мноогобразие, гомеоморфное $U$.
            \begin{proof}
                \[X \overset{g}{\to} Y\]
                График $\Gamma_g = \{(x, g(x)) \mid x \in X\}$
                \[U \overset{g}{\to} \Gamma_g \ - C^r \text{ отображение}\]
                \[x \overset{g}{\to} (x, g(x)) \overset{g^-1 - \text{проекция на X}}{\to} X\]
            \end{proof}
        \end{lemma*}
        По лемме теорема доказана.
    \end{proof}
\end{theorem*}

\begin{note}
    Если у градиента функции в точке $p$ хотя бы одна координата не равна $0$, то $p$ - регулярная.
\end{note}

\begin{statement*}
    $X \overset{C^r}{\cong} Y$ - если $X$ - $C^r$-многобразие, то $Y$ - тоже.
    \begin{proof}
        Пусть $\psi: X \to Y$ - $C^r$-изоморфизм. Пусть $q\in Y, p = \psi^{-1}(q)$, по условию существует открытая
        окрестность $V$ точки $p$, $C^r$-изоморфная $U \open \mathbb{R}^k$. 
        \[U \overset{\varphi}{\cong} V \overset{\psi}{\cong} \psi(V) \open Y\]
        $\psi(V)$ - прообраз $V$ под действием $\psi^{-1}, V = \psi^{-1}(\psi(V))$\\
        $\psi \circ \varphi$ - $C^r$-изоморфизм $U$ на окрестность точки $q$ в $X$.
    \end{proof}
\end{statement*}

\begin{lemma*}[о локальном вложении]
    Пусть $f = \begin{array}{c}
        f_1(x_1, \hdots, x_k) \\
        \vdots \\
        f_n(x_1, \hdots, x_k)
    \end{array}$, $U \open \mathbb{R}^k, f: U \to \mathbb{R}^n$.\\
    Если точка $\overline{x_0} \in U$ такая, что $\frac{\partial f}{\partial x}(x_0) = k$, то у точки $x_0$ существует окрестность $\tilde{U}$ такая, что $f(\tilde{U})$ - $k$-мерное многобразие.

    \begin{proof}
        \[\begin{pmatrix}
            \frac{\partial f_1}{\partial x_1} & \hdots & \frac{\partial f_1}{\partial x_k} \\
            \vdots & & \vdots\\
            \frac{\partial f_k}{\partial x_1} & \hdots & \frac{\partial f_k}{\partial x_k} \\
            \vdots & & \vdots\\
            \frac{\partial f_n}{\partial x_1} & \hdots & \frac{\partial f_n}{\partial x_k}
        \end{pmatrix}\]
        Переставим $f$ (если надо) и считаем первые $k$ строк невырожденными в $x_0$.
        Рассмотрим урезанное отображение $\psi = (\begin{array}{c} f_1 \\ \vdots \\ f_k \end{array}): U \to \mathbb{R}^k$\\
        По теореме о локальной обратимости существует окрестность $\tilde{U} \ni x_0$, такая, что 
        $\psi(\tilde{U}) \open \mathbb{R}^k$ и $\psi|_{\tilde{U}}: \tilde{U} \to \psi(\tilde{U})$ - $C^r$-изоморфизм.
        \[\psi(\tilde{U}) \overset{\psi^{-1}}{\to} \tilde{U}\]
        $\tilde{U}$ в свою очередь, отображается в $f(\tilde{U})$ данным отображением:
        \[\begin{array}{c}
            f_1(\tilde{x_1}, \hdots, \tilde{x_k}) \\
            \vdots \\
            f_k(\tilde{x_1}, \hdots, \tilde{x_k}) \\
            f_{\text{остальные индексы}}(\tilde{x_1}, \hdots, \tilde{x_k})
        \end{array}\]
        Причем $f_1, \hdots, f_k = (x_1, \hdots, x_k)$. 
        \[f_{\text{остальные}}(\tilde{x_1}, \hdots, \tilde{x_k}) = f_{\text{остальные}}\circ \psi(x_1, \hdots, x_k)\]
        $f(\tilde{U})= \{(\overline{x}, \psi(\overline{x})) \mid \overline{x} \in \tilde{U}\}$ - график отображения $\psi$, то есть по соответствующей лемме о графике это мноогобразие.
    \end{proof}
\end{lemma*}