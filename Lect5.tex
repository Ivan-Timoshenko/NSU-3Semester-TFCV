\section{Конформные отображения}
    Рассмотрим функцию $w = f(z) = u(x,y) + iv(x,y)$, она аналитична в области $D$,
    причем  $z_o \in D, f'(z_0) \neq 0$.
    \[\text{Условие Коши-Римана: } \begin{cases*}
        u_x = v_y \\ v_x = - u_y
    \end{cases*}\]
    Якобиан $\frac{D(u, v)}{D(x, y)} = \left| \begin{array}{cc}
        u_x & u_y \\ v_x & v_y
    \end{array}\right| = u_x^2 + u_y^2 = \left| f'(z)^2\right| \overset{f'(z_0)\neq 0}{\implies} \frac{D(u,v)}{D(x,y)}|_{z=z_0} \neq 0$.\\
    Значит по теореме о неявной функции система $\begin{cases*}
        u = u(x,y) \\ v = v(x,y)
    \end{cases*}$ в некоторой окрестности точки $w_0 = f(z_0)$ определяет однозначные непрерывные функции $\begin{cases*}
        x = x(u,v) \\ y = y(u, v)
    \end{cases*}$, значения которых лежат в окрестности точки $z_0$.
    \[(f^{-1}(w_0))' = \lim_{\Delta w \to 0}\frac{\Delta z}{\Delta w} = \lim \frac{1}{\frac{\Delta w}{\Delta z}} = \frac{1}{f'(z_0)}\]
    Обратная функция $z = f^{-1}(w)$ непрерывна в окрестности точки $w_0 \implies (\Delta w \to 0) \iff (\Delta z \to 0)$
    
\begin{definition}
    Если $f'(z) \neq 0 \ \forall z \in D$, то функция $f(z)$ называется локально однолистной в $D$ (синоним "взаимнооднозначной").
\end{definition}

\begin{note}
    Из локальной обратимости не следует глобальная обратимость (функция $w = f(z)$ локально обратима в области $D$ $\nRightarrow$ функция $w = f(z)$ однолистна в $D$).
\end{note}

\subsection{Геометрический смысл $\left|f'(z)\right|$ и $\arg f'(z)$ аналитической функции $f(z)$}
Пусть функция $w = f(z) = u(x,y) + iv(x,y)$ - аналитическая функция в области $D$ и $f'(z) \neq 0$. \\
Пусть $\gamma$ - гладкая кривая Жордана, заданная уравнением $z = z(t) = x(t) + iy(t)$, где $t \in [\alpha, \beta] \subset \mathbb{R}$. Если $z_0 = z(t_0), \ t_0 \in [\alpha, \beta]$, то пусть $z'(t_0) \neq 0$. \\
Функция $w = f(z)$ отображает кривую $\gamma$ в $\Gamma$, $w_0 = f(z_0) \in \Gamma$.\\
Уравнение кривой $\Gamma$:
\begin{equation}
    \label{gamma_equation}
    \Gamma: w = w(t) = f(z(t)) = u(x(t), y(t)) + iv(x(t), y(t)) \implies w'(t_0) = f'(z_0)\cdot z'(t_0)
\end{equation}

\begin{equation}
    \label{diffs1}
    \begin{cases*}
        dz = z'(t)dt = \left[x'(t) + iy'(t)\right]dt \\
        dw = w'(t)dt = \left[u'(t) + iv'(t)\right]dt
    \end{cases*}
\end{equation}

\begin{equation}
    \begin{cases*}
        ds = \sqrt{x'(t)^2 + y'(t)^2}dt = \left|z'(t)\right|dt \\
        d\sigma = \sqrt{u'(t)^2 + v'(t)^2}dt = \left|w'(t)\right|dt
    \end{cases*} \text{ - элементы длины кривой $\gamma$(ds) и $\Gamma(d\sigma)$ соответственно}
\end{equation}

Из \eqref{diffs1} $\implies \frac{dw}{dz}|_{t=t_0} \overset{\eqref{gamma_equation}}{=}f'(z_0) \implies \left|f'(z_0)\right| = \frac{d\sigma_0}{ds_0}$, где $d\sigma_0$ и $ds_0$ - элементы длины кривых
$\gamma$ и $\Gamma$ в точках $z_0 = z(t_0)$ и $w_0 = w(t_0) \implies \left| f'(z_0)\right|$ - коэффициент искажения элемента длины дуги
в точке $z_0$ при отображении $w = f(z)$. \\
Коэффициент искажения не зависит от направления кривой в этой точке, а лишь от $f$.\\
Будем говорить, что при отображении $\begin{cases*}
    w = f(z) \\ f'(z_0) \neq 0
\end{cases*}$ в $z_0$ имеет место \textbf{постоянство искажения}.

Из \eqref{gamma_equation}:
\begin{equation}
    \label{angle_constancy}
    \implies \arg f'(z_0) = \arg w'(t_0) - \arg z'(t_0)
\end{equation}
Из \eqref{angle_constancy} $\arg f'(z_0)$ - угол поворота $\forall$ кривой в точке $z_0$ при отображении $w=f(z)$.

\begin{definition}
    Пусть функция $w=f(z)$ аналитична в $D$. Если $f'(z) \neq 0$ для $z \in D$, то в $z$ при отображении $w = f(z)$
    имеет место \textbf{консерватизм углов}.
\end{definition}


\begin{definition}
    Конформное отображение области $D$ - это топологическое (гомеоморфное: взаимнооднозначное и непрерывное) 
    отображение этой облсти, при котором в каждой точке $z \in D$ имеет место консерватизм углов и постоянство искажения.
\end{definition}

Из геометрического смысла $\left| f'(z_0)\right|$ и $\arg f'(z_0)$ делаем вывод:
\begin{statement*}
    Если функция $w = f(z)$ осуществляет отображение области $D$ и $\forall z \in D$ имеет место консерватизм углов и постоянство искажения, то $w = f(z)$
    аналитична в $D$ и $f'(z) \neq 0 \ \forall z \in D$.
\end{statement*}
Из него сразу вытекает:
\begin{statement*}
    Конформное отображение области $D$ осуществляется однолистной в $D$ аналитической функцией $w = f(z), \ f'(z) \neq 0$.
\end{statement*}  

\subsection{Дробно-линейное отображение}
\begin{definition}
    Дробно-линейным отображением (ДЛО) называется отображение $w(z) = \frac{az + b}{cz + d}$, у которого $ad-bc \neq 0$. Если $ad-bc = 0$, то такое отображение называется \textbf{вырожденным} дробно-линейным отображением.
\end{definition}

Пусть отображение $w = \frac{az + b}{cz + d}$ переводит $z_1$ в $w_1$, $z_2$ - в $w_2$:
\[w_1 = \frac{az_1 + b}{cz_1 + d}, \ w_2 = \frac{az_2 + b}{cz_2 + d}, \ w_1-w_2 = \frac{(ad-bc)(z_1-z_2)}{(cz_1 + d)(cz_2 + d)}\]

\begin{definition}
    Линейное отображение: $w = az+b$ можно представить как композицию отображений следующих отображений:
    \[\zeta = \left| a \right| z \text{(подобие)}, \ v = e^{i\arg a}\cdot \zeta \text{(поворот на угол $\arg a$ вокруг O)}, w = v+b \text{(сдвиг)}\]
    Таким образом линейное отображение конформно отображает комплексную плоскость \textcircled{$z$} на комплексную плоскость \textcircled{$w$}:
    \[w(z): \overline{\mathbb{C}} \to \overline{\mathbb{C}}\]
\end{definition}
\par
Покажем, что $w = \frac{1}{z}$ обладает круговым свойством.
Функция $w = \frac{1}{z}$ однолистна и аналитична в $\overline{\mathbb{C}}, \ w' \neq 0 \ \forall z \in \overline{\mathbb{C}}$, 
значит $w = \frac{1}{z}$ конформно отображает комплексную плоскость \textcircled{$z$} на комплексную плоскость \textcircled{$w$}: $w(z)=\frac{1}{z}:\overline{\mathbb{C}} \overset{\text{конформно}}{\underset{\text{на}}{\to}} \overline{\mathbb{C}}$. \\
Любая обобщенная окружность на $\overline{\Compl}$ имеет уравнение вида
\[A(x^2+y^2) + bx + b_1y + c = 0\]
Можем преобразовать:
\[z = x + iy \implies Az \cdot \overline{z} + \overline{B}z \cdot B\overline{z} + c = 0, \ B = \frac{b + ib_1}{2}\]
Подействуем на эту окружность отображением $w = \frac{1}{z}$:
\[A \cdot \frac{1}{w \overline{w}} + \overline{B}\cdot \frac{1}{w} + B\cdot \frac{1}{\overline{w}} + c = 0\]
Значит уравнение обобщенной окружности на плоскости \textcircled{w} имеет вид:
\[C w \overline{w} + \overline{B} \overline{w} + Bw + A = 0\]
То есть функция $w = \frac{1}{z}$ обладает круговым свойством.

\par
Любое невырожденное ДЛО $w = \frac{az+b}{cz+d}$ либо является линейным (при $c = 0$), либо композиция следующих отображений:
\[\zeta = \frac{c^2}{ad-bc}z  +\frac{cd}{ad-bc}, \ v = \frac{1}{\zeta}, \ w = v + \frac{\alpha}{c}\]
А значит любое непрерывное ДЛО обладает круговым свойством и конформно отображает комплексную плоскость \textcircled{$z$} на \textcircled{$w$}.


\subsection{Симметрия относительно окружности $\left| z - z_0 = R \right|$}
\begin{definition}
    Точки $z_1, z_2$ называются симметричным относительно окружности $\left| z - z_0 = R \right|$, если они лежат на одном луче
    с началом в $z_0$ и $\left|z_1 - z_0\right|\cdot \left| z_2 - z_0 \right| = R$. Отображение $w = z_0 + \frac{R^2}{\overline{z} - \overline{z_0}}$ меняет внутреннюю и внешнюю области окружности местами, оставляя окружность неподвижной.
\end{definition}

\subsection{Симметрия относительно прямой $z = z_0 + te^{i\varphi}, t \in \Real$}
Две точки симметричны относительно прямой на комплексной плоскости, если перпендикуляры, опущенные из этих точек на прямую равны по длине.
Отображение $w = z_0 + e^{2i\varphi}(z-\overline{z_0})$ отображает комплексную плоскость симметрично относительно прямой (меняет две области по ее сторонам местами).

\subsection{Сдвиг плоскости на вектор}
Сдвиг плоскости на вектор $b$ - это композиция двух симметрий:
\begin{enumerate}
    \item $\zeta = e^{2i\varphi}\overline{z}$
    \item $w = \frac{b}{2} + e^{2i\varphi}(\overline{\zeta} - \frac{\overline{b}}{2}), \text{ где }\varphi = \frac{\pi}{2} - \arg b$
\end{enumerate}


\subsection{Подобие}
Отображение подобия $w = kz, \ k > 0$ - композиция двух симметрий:
\begin{enumerate}
    \item $\zeta = \frac{1}{z} \text{ (относительно окружности $\left| z \right| = 1$)}$
    \item $w = \frac{k}{\zeta} \text{ (относительно окружности $\left| \zeta  \right| = \sqrt{k}$)}$
\end{enumerate}

\begin{statement*}
    Любое невырожденное ДЛО $w = \frac{az+b}{cz + d}$ - это композиция четного числа отображений симметрии отностельно обобщенных окружностей.
\end{statement*}

\begin{note}
    \[w = \frac{az+b}{cz+d}, \ \omega = \frac{\alpha w + \beta}{\gamma w + \delta} = \frac{Az + B}{Cz + D} \text{ - если $AD - BC = (ad-bc)(\alpha \delta - \gamma \beta)$}. \]
    Композиция двух ДЛО является невырожденным ДЛО $\iff$ эти два ДЛО невырожденные.
\end{note}

\begin{statement*}[Следствие]
    Композиция любого четного числа симметрий относительно обобщенных окружностей является ДЛО.
\end{statement*}

\begin{theorem*}
    Пусть заданы 3 точки: $z_1, z_2, z_3 \in \overline{\Compl}$, и 3 точки $w_1, w_2, w_3 \in \overline{\Compl}$. Тогда существует единственное ДЛО $w = \frac{az+b}{cz+d}$ такое, что 
    \[w_1 = w(z_1), \  w_2 = w(z_2), \ w_3 = w(z_3)\]
    \begin{proof}
        \[w_j - w_k = \frac{(ad-bc)(z_j - z_k)}{(cz_j + d)(cz_k + d)}\]
        Причем 
        \begin{equation}\label{unharmonic} \frac{w-w_1}{w-w_2} \cdot \frac{w_3 - w_2}{w_3 - w_1} = \frac{z-z_1}{z - z_2} \cdot \frac{z_3 - z_2}{z_3 - z_1}\end{equation}
        Используя $w = \frac{az+b}{cz+d}$ что-то куда-то подставим и все будет хорошо. Соотношение \eqref{unharmonic} называются ангармоническим соотношением четверки точек.
    \end{proof}
\end{theorem*}

\begin{statement*}[Следствие]
    Любое невырожденное ДЛО сохраняет ангармоническое соотношение.
\end{statement*}


\begin{definition}
    Положительное направление обход границы это такое направление, при котором область $D$ остается слева.
\end{definition}

\begin{statement*}
    Любое невырожденное ДЛО сохраняет положительное направление обхода (то есть ориентацию на $\Compl$).
\end{statement*}

\par
Общий вид ДЛО: $w = \frac{az+b}{cz+d}$, оно отображает область $Im(z) > 0$ на единичную окружность $\left| z \right| < 1$.
Прообразом $z_1$ точки $w_1 = 0$ будет точка $-\frac{b}{a}$, прообразом точки $w_2 = \infty$ будет точка $z_2 = -\frac{d}{c}$. \\
Так как точки $w_1 = 0; \ w_2 = \infty$ симметричны относительно окружности $\left| w \right| < 1$, то их прообразы $z_1$ и $z_2$ должны быть симметрично относительно прямой $y=0$. Обозначим $z_0 = z_1 = -\frac{b}{a}$ и 
$\overline{z_0} = z_2 = -\frac{d}{c}$. \\
Раз $w_1 \in w(D)$, то $z_0 \in D \implies Im(z_0) > 0$.
\[w = \frac{az+b}{cz+d} = \frac{a}{c}\cdot \frac{z + \frac{b}{a}}{z + \frac{d}{c}} = \frac{a}{c}\cdot \frac{z-z_0}{z - \overline{z_0}} \implies \left| w \right| = \left| \frac{a}{c} \right|\cdot \left| \frac{z - z_0}{z - \overline{z_0}} \right|\]
При $y = 0$: $1 = \left| w \right| = \left| \frac{a}{c} \right| = 1$, так как $\left| \frac{a}{c} \right|$ - это константа, равная единице, то эта точка лежит на единичной окружности, 
тогда ее представление в формуле Эйлера $\frac{a}{c} = e^{i\varphi}$ для некоторого $\varphi$. В таком случае $w = e^{i\varphi} \cdot \frac{z-z_0}{z - \overline{z_0}}$, где $\varphi$ - константа $\in [0 ; 2\pi)$, а $z_0$ - константа $\in \Compl$, но $Im(z_0) > 0$.\\
$w = e^{i\varphi}\cdot \frac{z-z_0}{z - \overline{z_0}}$ - этой общий вид ДЛО, которое отображает верхнюю полуплоскость $Im(z) > 0$ на единичный круг.\\
В случае, когда $Im(z_0) < 0$ - верхняя полуплоскость отображает во внешность единичной окружности $\left| z \right| > 1$.

\par
Построим ДЛО для отображения внутренности единичной окружности $\left| z  \right| < 1$ в такую же единичную окружность $\left| w \right| < 1$.
\[\begin{cases*}
    z_1 = -\frac{b}{a} \\ z_2 = -\frac{d}{c}
\end{cases*} \implies 
\begin{cases*}
    w_1 = 0 \\ w_2 = \infty
\end{cases*}\]
Аналогично предыдущим рассуждениям $w_1 = 0, \ w_2 = \infty$ симметричны относительно $\left| w \right| = 1$, значит их прообразы симметричны 
относительно $\left| z \right| = 1$. В таком случае $z_1 = \frac{1}{\overline{z_1}}$. Обозначим $z_0 = z_1 = -\frac{b}{a} \implies -\frac{d}{c} = \frac{1}{\overline{z_0}}$. При этом $\left| z_0 \right| < 1$ так как $z_0 \in \left\{ \left| z \right| < 1 \right\}$.
\[w = \frac{az+b}{cz+d} = \frac{a}{c} \cdot \frac{z + \frac{b}{a}}{z + \frac{d}{c}} = \frac{a}{c}\cdot \frac{z - z_0}{z - \overline{z_0}} = \frac{a\overline{z_0}}{cz_0} \cdot \frac{z - z}{\overline{z_0} - \frac{1}{z}} \implies \left| w \right| = \left| \frac{a\overline{z_0}}{cz} \right| \cdot \left| \frac{z - z_0}{\overline{z_0} - \frac{1}{z}} \right| \]
На границе окружности при $\left| z \right| = 1$:
\[\left| z \right| = 1 \to 1 = \left| w \right| = \left| \frac{a\overline{z_0}}{c}\right| \cdot \left| \frac{z - z_0}{\overline{z_0} - \overline{z}} \right| \implies \left| \frac{a\overline{z_0}}{c} \right| - const\]
То есть $\left| \frac{a\overline{z_0}}{c} \right| = e^{i\varphi}$ для постоянного $\varphi \in [0, 2\pi)$.\\
Значит $w = e^{i\varphi}\cdot \frac{z - z_0}{\overline{z_0}z -1}$ - общий вид ДЛО, отображающего $\left| z \right| < 1$ в $\left| w \right|< 1$.
Если $\left| z_0 \right| > 1$, то $\left| z \right| < 1$ перейдет в $\left| w \right| > 1$. 


