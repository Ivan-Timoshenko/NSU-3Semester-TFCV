\section{Конформные отображения}
    Рассмотрим функцию $w = f(z) = u(x,y) + iv(x,y)$, она аналитична в области $D$,
    причем  $z_o \in D, f'(z_0) \neq 0$.
    \[\text{Условие Коши-Римана: } \begin{cases*}
        u_x = v_y \\ v_x = - u_y
    \end{cases*}\]
    Якобиан $\frac{D(u, v)}{D(x, y)} = \left| \begin{array}{cc}
        u_x & u_y \\ v_x & v_y
    \end{array}\right| = u_x^2 + u_y^2 = \left| f'(z)^2\right| \overset{f'(z_0)\neq 0}{\implies} \frac{D(u,v)}{D(x,y)}|_{z=z_0} \neq 0$.\\
    Значит по теореме о неявной функции система $\begin{cases*}
        u = u(x,y) \\ v = v(x,y)
    \end{cases*}$ в некоторой окрестности точки $w_0 = f(z_0)$ определяет однозначные непрерывные функции $\begin{cases*}
        x = x(u,v) \\ y = y(u, v)
    \end{cases*}$, значения которых лежат в окрестности точки $z_0$.
    \[(f^{-1}(w_0))' = \lim_{\Delta w \to 0}\frac{\Delta z}{\Delta w} = \lim \frac{1}{\frac{\Delta w}{\Delta z}} = \frac{1}{f'(z_0)}\]
    Обратная функция $z = f^{-1}(w)$ непрерывна в окрестности точки $w_0 \implies (\Delta w \to 0) \iff (\Delta z \to 0)$
    
\begin{definition}
    Если $f'(z) \neq 0 \ \forall z \in D$, то функция $f(z)$ называется локально однолистной в $D$ (синоним "взаимнооднозначной").
\end{definition}

\begin{note}
    Из локальной обратимости не следует глобальная обратимость (функция $w = f(z)$ локально обратима в области $D$ $\nRightarrow$ функция $w = f(z)$ однолистна в $D$).
\end{note}

\subsection{Геометрический смысл $\left|f'(z)\right|$ и $\arg f'(z)$ аналитической функции $f(z)$}
Пусть функция $w = f(z) = u(x,y) + iv(x,y)$ - аналитическая функция в области $D$ и $f'(z) \neq 0$. \\
Пусть $\gamma$ - гладкая кривая Жордана, заданная уравнением $z = z(t) = x(t) + iy(t)$, где $t \in [\alpha, \beta] \subset \mathbb{R}$. Если $z_0 = z(t_0), \ t_0 \in [\alpha, \beta]$, то пусть $z'(t_0) \neq 0$. \\
Функция $w = f(z)$ отображает кривую $\gamma$ в $\Gamma$, $w_0 = f(z_0) \in \Gamma$.\\
Уравнение кривой $\Gamma$:
\begin{equation}
    \label{gamma_equation}
    \Gamma: w = w(t) = f(z(t)) = u(x(t), y(t)) + iv(x(t), y(t)) \implies w'(t_0) = f'(z_0)\cdot z'(t_0)
\end{equation}

\begin{equation}
    \label{diffs1}
    \begin{cases*}
        dz = z'(t)dt = \left[x'(t) + iy'(t)\right]dt \\
        dw = w'(t)dt = \left[u'(t) + iv'(t)\right]dt
    \end{cases*}
\end{equation}

\begin{equation}
    \begin{cases*}
        ds = \sqrt{x'(t)^2 + y'(t)^2}dt = \left|z'(t)\right|dt \\
        d\sigma = \sqrt{u'(t)^2 + v'(t)^2}dt = \left|w'(t)\right|dt
    \end{cases*} \text{ - элементы длины кривой $\gamma$(ds) и $\Gamma(d\sigma)$ соответственно}
\end{equation}

Из \eqref{diffs1} $\implies \frac{dw}{dz}|_{t=t_0} \overset{\eqref{gamma_equation}}{=}f'(z_0) \implies \left|f'(z_0)\right| = \frac{d\sigma_0}{ds_0}$, где $d\sigma_0$ и $ds_0$ - элементы длины кривых
$\gamma$ и $\Gamma$ в точках $z_0 = z(t_0)$ и $w_0 = w(t_0) \implies \left| f'(z_0)\right|$ - коэффициент искажения элемента длины дуги
в точке $z_0$ при отображении $w = f(z)$. \\
Коэффициент искажения не зависит от направления кривой в этой точке, а лишь от $f$.\\
Будем говорить, что при отображении $\begin{cases*}
    w = f(z) \\ f'(z_0) \neq 0
\end{cases*}$ в $z_0$ имеет место \textbf{постоянство искажения}.

Из \eqref{gamma_equation}:
\begin{equation}
    \label{angle_constancy}
    \implies \arg f'(z_0) = \arg w'(t_0) - \arg z'(t_0)
\end{equation}
Из \eqref{angle_constancy} $\arg f'(z_0)$ - угол поворота $\forall$ кривой в точке $z_0$ при отображении $w=f(z)$.

\begin{definition}
    Пусть функция $w=f(z)$ аналитична в $D$. Если $f'(z) \neq 0$ для $z \in D$, то в $z$ при отображении $w = f(z)$
    имеет место \textbf{консерватизм углов}.
\end{definition}


\begin{definition}
    Конформное отображение области $D$ - это топологическое (гомеоморфное: взаимнооднозначное и непрерывное) 
    отображение этой облсти, при котором в каждой точке $z \in D$ имеет место консерватизм углов и постоянство искажения.
\end{definition}

Из геометрического смысла $\left| f'(z_0)\right|$ и $\arg f'(z_0)$ делаем вывод:
\begin{statement*}
    Если функция $w = f(z)$ осуществляет отображение области $D$ и $\forall z \in D$ имеет место консерватизм углов и постоянство искажения, то $w = f(z)$
    аналитична в $D$ и $f'(z) \neq 0 \ \forall z \in D$.
\end{statement*}
Из него сразу вытекает:
\begin{statement*}
    Конформное отображение области $D$ осуществляется однолистной в $D$ аналитической функцией $w = f(z), \ f'(z) \neq 0$.
\end{statement*}  
