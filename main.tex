\documentclass[a4paper]{article}

\usepackage{cmap}
\usepackage[T2A]{fontenc}
\usepackage[utf8]{inputenc}
\usepackage[english,russian]{babel}
\usepackage{amsthm}
\usepackage{amssymb}
\usepackage{amsmath}
\usepackage{mathtools}
\usepackage{indentfirst}
\usepackage{fullpage}
\usepackage{titlesec}
\usepackage{multicol}
\usepackage{parskip}
\usepackage{graphicx}
\usepackage{tikz}
\usepackage{wrapfig}
\usepackage{breqn}

\newcommand{\open}{\underset{op}{\subset}}

\newtheoremstyle{definition}{3pt}{3pt}{\upshape}{}{\bfseries}{.}{.5em}{}
\theoremstyle{definition}
\newtheorem{definition}{Опр.}

\newtheoremstyle{statement}{3pt}{3pt}{\upshape}{}{\bfseries}{}{.5em}{}
\theoremstyle{statement}
\newtheorem{statement}{Утв.}
\newtheorem*{statement*}{Утв.}

\newtheoremstyle{lemma}{3pt}{3pt}{\upshape}{}{\bfseries}{}{.5em}{}
\theoremstyle{lemma}
\newtheorem{lemma}{Лемма}
\newtheorem*{lemma*}{Лемма}

\newtheoremstyle{note}{3pt}{3pt}{\upshape}{}{\bfseries}{:}{.5em}{}
\theoremstyle{note}
\newtheorem*{note}{Замечание}

\newtheorem{theorem}{Теорема}
\newtheorem*{theorem*}{Теорема}

\newtheoremstyle{example}{3pt}{3pt}{\upshape}{}{\bfseries}{:}{.5em}{}
\theoremstyle{example}
\newtheorem*{example}{Пример}

\title{Лекции по математическому анализу, 3 семестр}
\date{}
\author{Тимошенко Иван, 24123}


\begin{document}
    \maketitle
    \section{Дифференциирование функций}

    
\begin{definition}
    Функция $f(x)$ дифференциируема в точке $p \in U$, если:
    \begin{enumerate}
        \item $f$  определена в некоторой окрестности точки $p$ ($p \in Int(U)$)
        \item $\exists \lim_{\Delta h \to 0} \frac{f(p + \Delta h) - f(p)}{\Delta h} \in \mathbb{R} $ (и этот предел равен $f'(p)$)
    \end{enumerate}
\end{definition}

\subsection{Экстремумы}
\medskip
\textbf{Необходимое условие экстремума:}\\
Пусть $f \in D(p)$ (дифференциируема в p). Если p - экстремум, то $f'(p) = 0$.

\begin{note}
    НО например для $f(x) = x^{3} \quad f'(0) = 0$, но $f(x)$ не дифференциируема в 0.
\end{note}
\begin{note}
Необходимое условие экстремума выполнено лишь для точек во внутренности области определения,
точки на границе необходимо проверять отдельно.
\end{note}


\textbf{Достаточное условие экстремума:}\\
Пусть $f \in D^{2}(p)$ (дважды дифференциируема в $p$) и $f'(p) = 0$.В таком случае если
\begin{itemize}
    \item $f''(p) < 0$ - точка $p$ является локальным максимумом и экстремумом.
    \item $f''(p) > 0$ - точка $p$ является локальным минимумом и экстремумом.
\end{itemize} 

\medskip
Пусть $f:U \in \mathbb{R}^{n} \to \mathbb{R}^{k}$, $p \in U$. 
Функция $f$ дифференциируема в $p$, если:
\begin{enumerate}
    \item $p \in Int(U) \quad
        (\exists \epsilon > 0 \quad B_\epsilon(p) \subset U)$
    \item $\exists$ дифференциал функции (линейное отображение)
    $f$ в точке $p$ \quad $df(p): \mathbb{R}^{n} \to \mathbb{R}^{k}$
    такое, что    \[f(x) = f(p) + df(p)<x-p> + \alpha(x) \quad (\alpha(x) \underset{x \to p}{=} o(x-p)) \]
    При сдвиге точки $p$ на вектор $h$: \[f(p+h) = f(p) + df(p)<h> + o(|h|)\]
\end{enumerate}

\subsection{Частные производные}
\medskip
Стандартный контекст в котором работаем:
\[f:U \subset \mathbb{R}^{n} \to \mathbb{R}^{k}, \quad p \in U, \quad p = (p_1, p_2, \dots, p_n) \]

\begin{definition}
    Частная производная по координате $x_i$ это:
    \[\frac{\partial f}{\partial x_i}(p) = \lim_{t \to 0} \frac{f(p_1,.., p_i + t,.., p_n) - f(p_1,\ldots,p_n)}{t}\]
\end{definition}

Пример для $f(x, y) = x^y$:
\[\frac{\partial f}{\partial x}(x, y) = yx^{y-1}; \quad \frac{\partial f}{\partial y}(x, y) = x^y\ln(x)\]

\begin{definition}
    Прозводная вдоль вектора $v$:
    \[\frac{\partial f}{\partial v}(p) := \lim_{t \to 0} \frac{f(p + tv) - f(p)}{t}\]
    Если $v = e_i = (0, \ldots, 0, \underset{i}{1}, 0, \ldots, 0)$, то $\frac{\partial f}{\partial v} = \frac{\partial f}{\partial x_i} = f'_{x_i}$
\end{definition}

Пример:
\[f(x) = \begin{cases}
    \frac{x^2y}{x^4 + y^2} \quad $при$(x, y) \neq (0, 0) \\
    0 \quad $при$ (x, y) = (0,0)
\end{cases}\]
По любому вектору $v = (v_1, v_2)$ у функции есть производная в $(0, 0)$:
\[\lim_{t \to 0} \frac{f(tv) - f(0, 0)}{t} = \frac{t^3v_1^2v^2}{t^5v_1^4 + t^3v_2^2} = \frac{v_1^2v_2}{t^2v_1^2 v_2^2} 
    \underset{t \to 0}{=} 
    \begin{cases}
        0 \quad v_2 = 0\\
        \frac{v_1^2}{v_2} \quad v_2 \neq 0
    \end{cases}\]

\begin{statement}
    Если $f$ дифференциируема в $p$, то $f$ - непрерына в $p$.
\end{statement}

\begin{proof}
    \[f(p+h) - f(p) \underset{h \to 0}{=} df(p)<h>\] 
    Линейное отображение df(p)<> непрерывно, $o(h) \underset{h \to 0}{\to} 0$,
    т.е. $f(p+h) \underset{h \to 0}{\to} f(p)$.
\end{proof}

\textbf{Достаточный признак дифференциируемости:}\\
Все частные производные непрерырвны в $p$ ($f \in D(p)$).

Пример:
$f(x, y) = x^y$ дифференциируема во всех точках $(x_0, y_0)$, где $x_0 > 0$.
\[\frac{\partial f}{\partial x} = yx^{y-1} \qquad \frac{\partial f}{\partial y} = x^y\ln(x)\]
Частные производные непрерывны, значит и функция непрерывна.

\subsection{Матрица Якоби и градиент функции}
Контекст в котором работаем:
\[f:U \subset \mathbb{R}^{n} \to \mathbb{R}^{k}, \quad p \in U\]
\begin{definition}
Матрицей Якоби называют матрицу
\[D_f(p) = 
\begin{pmatrix}
    \frac{\partial f_1}{\partial x_1} & \ldots & \frac{\partial f_1}{\partial x_n}\\
    \vdots & \ddots & \vdots \\
    \frac{\partial f_k}{\partial x_1} & \dots & \frac{\partial f_k}{\partial x_n}
\end{pmatrix}\]
\end{definition}

В случае, если функция $f$ отображает $\mathbb{R}^n \to \mathbb{R}$, то матрица Якоби принимает вид $1 \times n$ и называется \textbf{градиентом функции}.
\begin{definition}
    Градиентом функции называется вектор 
    \[D_f = 
    \begin{pmatrix}
        \frac{\partial f}{\partial x_1}, & \dots, & \frac{\partial f}{\partial x_n}
    \end{pmatrix}
    \]
\end{definition}
    \section{Множества на $\overline{\mathbb{C}}$}

\begin{definition}
    \begin{itemize}
        \item $\delta$ - окрестность точки $z_0 \in \mathbb{C}$ - это множество $C(\delta, z_0) = \{z \in \mathbb{C} \ | \ \left| z - z_0\right| < \delta\}, \delta > 0$.
        \item Проколотая окрестность точки $z_0 \in \mathbb{C}$ - это множество $C^*(\delta, z_0) = C(\delta, z_0) \backslash \{z_0\}$
        \item Окрестность точки \{$\infty$\} - множество $\{z \in \mathbb{C} \ | \ \left| z \right| > \delta\}$
        \item Точка $z$ - изолированная точка множества $E \subset \overline{\mathbb{C}}$, если $\exists \delta > 0: C(\delta, z_0) \cap E = \{z_0\}$
    \end{itemize}
\end{definition}

\begin{definition}
    Точка $z$ называется предельной точкой множества $E \subset \overline{\mathbb{C}}$, если $\forall \delta > 0 \ C^*(\delta, z) \cap E \neq \varnothing$
\end{definition}

\begin{definition}
    Точка $z$ называется внутренней точкой множества $E \subset \overline{\mathbb{C}}$, если $\exists \delta > 0 \ C(\delta, z) \subset E$
\end{definition}

\begin{definition}
    Точка $z$ называется внешней точкой множества $E \subset \overline{\mathbb{C}}$, если $\exists \delta > 0 \ C(\delta, z) \cap E = \varnothing$
\end{definition}

\begin{definition}
    Точка $z$ называется граничной точкой $E$, если 
    \[\forall \delta > 0 \begin{cases}
        C(\delta, z) \cap E \neq \varnothing \\
        C(\delta, z) \backslash E \neq \varnothing
    \end{cases}\]
\end{definition}

\begin{note}
    Если граничная точка $z \notin E$, то она является предельной для $E$.
\end{note}

\begin{definition}
    Граница $E$ - это совокупность всех граничных точек, обозначается $\partial E$.
\end{definition}

\begin{definition}
    Множестов называется ограниченным, если $\exists M \ (0 < M < \infty)$ - число, такое, что\\
    $\forall z \in E \ \left| z \right| < M$
\end{definition}

\begin{definition}
    Множество $E$ называется замкнутым, если оно содержит все свои граничные точки (или их нет).
\end{definition}

\begin{definition}
    Множество $E$ называется открытым, если $\forall z \in E$ $z$ является внутренней точкой для $E$, то есть 
    $\forall z \in E \ \exists \delta > 0: \ C(\delta, z) \subset E$
\end{definition}

\begin{definition}
    Замыкание множества $E$ это $\overline{E} = E \cup \partial E$.
\end{definition}

\begin{definition}
    Диаметр множества $d(E) = \underset{z, \xi \in E}{\sup} \left| z - \xi\right|$
\end{definition}

\begin{definition}
    Расстоянием между множествами $E$ и $G$ называется $D(E, G) = \underset{z \in E, \zeta \in G}{\inf} \left|z - \zeta\right|$
\end{definition}

\begin{definition}
    Множество $E$ называется связным, если его нельзя представить как $E = A \cup B, \ A, B \subset E$, таких, что 
    \begin{enumerate}
        \item $A, B \neq \varnothing$
        \item $A \cap B = \varnothing$
        \item $A$ и $B$ не содержат предельных точек друг друга.
    \end{enumerate}  
\end{definition}

\begin{definition}
    Множество $E$ называется линейно связным, если $\forall z_1, z_2 \in E \ \exists$ непрерывная функция $\varphi: [0, 1] \to E$ такая, что $\varphi(0) = z_1, \ \varphi(1) = z_2$
\end{definition}

\begin{definition}
    Областью называется открытое связное множество.
\end{definition}

\begin{definition}
    Компонента множества $E$ - $\forall$ максимальное по включению связное подмножество $E$.\\
    Область $E \neq \overline{\mathbb{C}}$ $n$-связная, если граница $\partial E$ состоит из $n$ компонент ($\overline{\mathbb{C}}$ считаем 1-связным).
\end{definition}

\begin{statement}
    $\forall$ множества $E \subset \overline{\mathbb{C}}$ граница $\partial E$ является замкнутым множеством.
    \begin{proof}
        Доказывается от противного: \\
        Допустим, $\exists$ предельная точка $z_0$ для $\partial E$: $z_0 \notin \partial E$.
    \end{proof}
\end{statement}

\begin{theorem}[Принцип Больцано-Вейерштрасса]
    У любого бесконечного множества $E \subset \overline{\mathbb{C}} \ \exists$ хотя бы одна предельная точка.
\end{theorem}

\begin{theorem}[Лемма Гейне-Бореля]
    Из бесконечного открытого покрытия замкнутого множества точек на $\overline{\mathbb{C}}$ можно выделить конечное подпокрытие.

    \textbf{Следствие:} На $\overline{\mathbb{C}}$ любое замкнутое и ограниченное множество является компактом.
\end{theorem}

\section{Предельные ряды комплексных чисел}

\begin{definition}
    Последовательность $\{z_n\}$ комплексных чисел $z_n = x_n + i y_n, \ n \in \mathbb{N}$ называется сходящейся к пределу $\alpha = a + ib$, если
    \[\forall \varepsilon > 0 \exists N: \left| z_n - \alpha \right| < \varepsilon \ \forall n > N\]
    Обозначается $\lim_{n \to \infty} z_n = \alpha$

    
\end{definition}
    
\textbf{Следствие:} $\lim_{n \to \infty} z_n = 0 \iff \lim_{n \to \infty} \left| z_n \right| = 0$
        
    
Попробуем перенести теорию последовательностей вещественных числе на комплексные числа:
\[\exists \lim_{n \to \infty} z_n = \alpha \implies \begin{cases}
    \exists \lim_{n \to \infty} x_n = a \\
    \exists \lim_{n \to \infty} y_n = b
\end{cases} \implies \] 
\[ \implies \begin{cases}
    \forall \varepsilon > 0 \ \exists N_1 \in \mathbb{N} \ \left| x_n - a \right| < \frac{\varepsilon}{2} \ \forall n > N_1 \\
    \forall \varepsilon > 0 \ \exists N_2 \in \mathbb{N} \ \left| y_n - b \right| < \frac{\varepsilon}{2} \ \forall n > N_2 
\end{cases} \implies \left| z_n - \alpha\right| \leq \left| x_n - a \right| + \left| y_n - b\right| < \varepsilon\]
Значит $\forall \varepsilon > 0 \ \exists N = \max(N_1, N_2) \ \left|z_n - \alpha\right| < \varepsilon \ \forall n > N$

Таким образом $\begin{cases}
    \exists \lim_{n \to \infty} x_n = a \\
    \exists \lim_{n \to \infty} y_n = b
\end{cases}  \implies \exists \lim_{n \to \infty} z_n = \alpha$
Это позволяет свести теорию последовательностей $\{z_n\}$ комплексных чисел к $\mathbb{R}$, 
то есть все теоремы о сходимости вещественных последовательностей переносятся на комплексные числа.

\begin{theorem*}[Критерий Коши]
    \[\text{Последовательность } \{z_n\} \text{ сходится} \iff \ \forall\varepsilon > 0 \ \exists N \in \mathbb{N}: \left| z_{n+m} - z_n \right|< \varepsilon  \text{ выполнено } \forall n, m > N\].
\end{theorem*}

\begin{definition}
    Последовательность $\{z_n\}$ сходится к $\infty$ ($\lim_{n \to \infty} z_n = \infty$) если
    \[\forall R > 0 \ \exists N \ \left| z_n \right| > R \text{ выполнено для } \forall n > N\]
    \[\text{При } z_n \neq 0 \ \lim_{n \to \infty}z_n = \infty \iff \lim_{n \to \infty} \frac{1}{z_n} = 0\]
\end{definition}

\newpage
\subsection{Ряды комплексных чисел}
Если у знака суммы ряда не указаны границы - считать, что слагаемые суммируются по индексу $k$, $k$ пробегает набор от 1 (иногда 0) до $\infty$
\begin{definition}
    Ряд комплексных чисел $\sum_{k=1}^{\infty}\alpha_k$ сходится (расходится), если сходится (расходится) последовательность частичных сумм $S_n = \sum_{k=1}^{n}\alpha_k$, т.е. $\exists \lim_{n \to \infty} S_n = S$.
\end{definition}

\begin{theorem*}[Критерий Коши сходимости ряда]
    \[\sum_{k=1}^{\infty} \text{ сходится } \iff \forall \varepsilon>0 \ \exists N \in \mathbb{N} \begin{cases*}
        \forall n > N \\
        \forall k \in \mathbb{N}
    \end{cases*} \left| S_{n+k} - S_n\right| < \varepsilon\]
\end{theorem*}

\begin{definition}
    Ряд $\alpha_k$ сходится абсолютно, если сходится ряд $\sum_{k = 1}^{\infty} \left| \alpha_k\right|$. Но поскольку $\left| \sum_{k = 1}^{m} \alpha_{n+k}\right| \leq \sum_{k=1}^{m}\left| \alpha_{n+k} \right|$, то из абсолютной сходимости ряда следует общая сходимость.
\end{definition}

\begin{statement}
    Член ряда $\alpha_k$ представим в виде $a_k + ib_k$. Тогда:
    \[\begin{cases*}
        \left| a_k \right| \leq \left|\alpha_k\right| \leq \left| a_k \right| + \left| b_k \right| \\
        \left| b_k \right| \leq \left|\alpha_k\right| \leq \left| a_k \right| + \left| b_k \right|
    \end{cases*}\]
    Значит, если $\sum_{k=1}^{\infty}\alpha_k$ абсолютно сходится, то $\sum_{k=1}^{\infty}a_k$ и  $\sum_{k=1}^{\infty}b_k$ тоже абсолютно сходятся. 
    В свою очередь, из абсолютной сходимости $\sum_{k=1}^{\infty}a_k$ и  $\sum_{k=1}^{\infty}b_k$ следует абсолютная сходимость $\sum_{k=1}^{\infty}\alpha_k$.
\end{statement}

\begin{definition}
    Произведение двух сходящихся рядов сходится к произведению их сумм.
\end{definition}

Стандартные разложения функций от комплексных чисел:
\begin{enumerate}
    \item $e^\alpha = 1 + \alpha + \frac{\alpha^2}{2!} + \hdots + \frac{\alpha^n}{n!} + \hdots = \sum_{k=0}^{\infty}\frac{\alpha^k}{k!}$
    \item $\sin(\alpha) = \alpha - \frac{\alpha^3}{3!} + \frac{\alpha^5}{5!} + \hdots = \sum_{k=0}^{\infty}\frac{(-1)^k\cdot\alpha^{2k+1}}{(2k+1)!}$
    \item $\cos(\alpha) = \sum_{k=0}^{\infty}\frac{(-1)^k\cdot\alpha^{2k}}{2k!}$
    \item $e^{i\alpha} = \sum\frac{(i\alpha)^k}{k!} = \sum\frac{(-1)^k\cdot\alpha^{2k}}{2k!} + i\sum\frac{(-1)^k\cdot\alpha^{(2k+1)}}{(2k+1)!} = \cos(\alpha) + i\sin(\alpha)$
\end{enumerate}

Пусть $\mid \alpha \mid = R, \ \arg(\alpha) = \varphi$. Тогда:
\[\alpha = R\cdot e^{i\cdot\arg(\alpha)} = R(\cos(\varphi) + i\sin(\varphi))\]
И поскольку можем представить $\alpha = a + ib$:
\[e^{\alpha} = e^a\cdot (\cos(b) + i\sin(b)), \quad \left| e^\alpha\right| = e^a, \ \arg(e^\alpha) = b\]
Поэтому $e^{2\pi k\cdot i} = 1$
    \textbf{Правило дифференциирования монома:\\}
\bigskip
Пусть $f(x)  = x_1^{i_1} \cdot x_2^{i_2} \cdot \dots \cdot x_m^{i_m}, \ x = (x_1, \dots, x_m)$. Тогда 
\[\frac{\partial^{i_1+i_2+ \dots + i_m} f}{\partial x_1^{i_1} \dots \partial x_m^{i_m}}(0) = i_1! \cdot \hdots \cdot i_m!\]
Любая другая производная любого порядка в точке $0$ равна $0$.

\subsection{Мульти-индексы}
Придумаем $\mu = (i_1, \hdots, i_m)$ - численный вектор, в котором $\forall j = \overline{1 \hdots m} \ i_j \geq 0$ и назовем его \textbf{мультииндексом}.
\par
Для мультииндексов определены операции:
\[\mu! := i_1! \cdot \hdots \cdot  i_m! \qquad \left| \mu \right| = \sum_{j = 1}^{m} i_j \text{ - порядок мультииндекса}\]
\[x \in \mathbb{R}^m, x = (x_1, \hdots, x_m), \quad x^{\mu} = x_1^{i_1} \cdot \hdots \cdot x_m^{i_m}\]
\[C^{\mu}_k = \frac{k!}{\mu!} = \frac{k!}{i_1!\cdot \hdots \cdot i_m!} \text{, где } k = \left|\mu\right|\]

\bigskip

Зададим контекст:
\[f:\mathbb{R}^m \to E, \ f \in D^k(p), \ \mu = (i_1, \hdots, i_m)\]
Тогда:
\[\frac{\partial^{\mu}f}{\partial x^{\mu}} = D^{\mu}f = \frac{\partial^{i_1}}{\partial x_1^{i_1}}(\frac{\partial^{i_2}}{\partial x_2^{i_2}}( \hdots (\frac{\partial^{i_m}}{\partial x_m^{i_m}}f)))\]

\begin{theorem*}[разложение Тейлора]
    $\exists!$ многочлен $A(x)$ степени $\leq k$ такой, что $f(x) - A(x) \underset{x \to p}{=} o(x-p)^k$    
    \[A(x) = f(p) + f'(p)(x-p) + \frac{f''(p)}{2!}(x-p)^2 + \hdots + \frac{f^{(k)}(p)}{k!}(x-p)^k\]
\end{theorem*}

\begin{theorem*}[разложение Тейлора для нескольких переменных]
    Пусть $f: \mathbb{R}^m \to E, \ f \in D^k(p)$, тогда $\exists!$ многочлен $A(x) \ deg(A) \leq k$,
    такой, что $f(x) - A(x) \underset{x \to p}{=} o(\left| x - p\right|^k)$:
    \[A(x) = f(p) + \frac{df(p) \langle x-p \rangle}{1!} + \frac{d^f(p)\langle x-p \rangle}{2!} + \hdots + \frac{d^kf(p)\langle x-p \rangle}{k!}\]
    \begin{proof}
        \textbf{Единственность:} Пусть есть два таких многочлена $A(x), B(x)$. Введем $C(x) := A(x) - B(x) = o(\left| x - p\right|^k)$. И докажем вспомогательное утверждение:
        \begin{statement*}
            $degC \leq k \ C(x) \underset{x \to p}{=} o(\left|x-p\right|^k)$, тогда $C \equiv 0$.
            \begin{proof}
                \begin{enumerate}
                    \item Фиксируем $v \in \mathbb{R}^m$ и рассмотрим $h(t) = C(p+tv)$ - многочлен одной переменной.
                    По условию $h(t) = o(t^k)$ для одной переменной (доказывали это в первом семестре), т.е. $h(t) \equiv 0$. В частности, при $t = 1 \ h(t) = C(p+v) = 0$.
                    \item Поскольку 1. выполняется $\forall v$, то $C(p+v) = 0 \  \forall v$.
                \end{enumerate}
            \end{proof}
            Тогда в силу доказанного утверждения получаем единственность.
        \end{statement*}
        \par
        \textbf{Существование:}
        Введем $g(x) = f(x) - A(x), \ f:\mathbb{R}^m \to E$. $g(p) = 0$ и все производные до порядка $k$
        включительно равны $0$ в $p$, $g \in D^K(p)$. Необходимо доказать, что из этого следует, что $g(x) = o(\left| x - p\right|^k)$.
        \\
        Пусть $\varepsilon > 0$, надо показать, что $\left| g(x) \right| < \varepsilon\cdot \left| x - p\right|^k$ в некоторой $U$ - окрестности точки $p$.
        Пусть $\varepsilon_{k-1}(x)$ - какая-то производная порядка $k-1$ функции $g$, $\varepsilon_{k-1}$ определена в некотором шаре $V_p$ с центром в $p$.
        \begin{equation}
            \label{eq1}
            \varepsilon_{k-1}(p) = 0, \ \frac{\partial \varepsilon_{k-1}}{\partial x_i}(p) = 0 \ \forall i = 1 \hdots m
        \end{equation}
        Поэтому имеется маленький шар $U \subset V_p$ в котором выполнено:
        \begin{equation}
            \label{eq2}
            \forall x \in U \ \left| \varepsilon_{k-1}\right|(x) \leq \varepsilon\cdot \left| x - p \right|
        \end{equation}
        В самом деле, $\varepsilon_{k-1}(x) = \varepsilon_{k-1}(p) + d\varepsilon_{k-1}(p)\langle x - p \rangle + o(\left| x - p \right|)$, причем первое слагаемое равно нулю из того, что "все производные до порядка $k$ включительно равны $0$ в $p$", а второе - из уравнения (\ref{eq}).
        Значит $\varepsilon_{k-1}(x) = o(\left| x - p\right|) \implies \varepsilon_{k-1}(x) \leq \varepsilon\left| x - p \right|$
        Итак, ясно, что существует шарик $U$, в котором все производные $k-1$ порядка 
        имеют оценку (\ref{eq2})
        Пусть $\varepsilon_{k-2}$ - какая-то производная функции $g$ порядка $k-2$. Все ее первые частные производные по доказанному в шаре $U$ оцениваются в $\varepsilon\cdot \left| x - p\right|$.
        По лемме о степенной оценке приращения для $\varepsilon_{k-2}$ выполнено в шаре $U$:
        \[\left| \varepsilon_{k-2}(x)\right| \leq \left| \frac{\varepsilon \left| x - p\right|^2}{2}\right|\]
        Для $k-3, k-4, \hdots$ аналогично. 
        \[\left| g(x) \right| = \left| g^{(k-k)}(x) \right| \leq \varepsilon \cdot \frac{\left| x- p \right| ^ k}{k!} \leq \varepsilon\cdot \left| x - p\right|^k
        \]
    \end{proof}
\end{theorem*}

\begin{theorem}[Достаточное условие локального экстремума функции многих переменных]    
    Пусть $f \in D^2(p), \ f:\mathbb{R}^m \to R$ и $df(p) > 0$. Тогда:
    \begin{enumerate}
        \item $d^2f(p) > 0$ - строгий локальный минимум
        \item $d^2f(p) < 0$ - строгий локальный максимум
        \item Если $d^2f(p)$ знаконеопределен, т.е. $\exists u \in \mathbb{R}^m \ d^2f(p)\langle u \rangle > 0$ и $\exists v \in \mathbb{R}^m \ d^2f(p)\langle v \rangle < 0$, то $p$ - седловая точка.
    \end{enumerate}
    \begin{proof}
        
        Докажем пункт 3:
        \begin{proof}
            Пусть $df(p) = 0$ и существуют вектора $u$ и $v$, такие, что $d^2f(p)\langle u \rangle > 0, \ d^2f(p)\langle v \rangle < 0$.
            
            Введем функцию $h(t) = f(p + tu)$. Тогда $h'(0) = df(p)\langle u \rangle = 0, \  h''(0) = d^2f(p)\langle u \rangle > 0$. Значит у функции $h$ в точке $0$ строгий минимум 
            (по достаточному условию экстремума для одной переменной). Аналогично вдоль $p + tv$ функция имеет строгий максимум, значит $p$ - седловая точка.
        \end{proof}
        
        Докажем пункт 1:
        \begin{proof}
            Пусть $d^2f(p) > 0$, то есть $\forall v \neq 0 \ d^f(p)\langle v \rangle > 0$. Сфера $S^{m-1} = \{v \in \mathbb{R}^m \quad \left| v \right| = 1\}$ - компактна (замкнута и ограничена). $d^2f(p):S^{m-1} \to \mathbb{R}$ - однородный многочлен второго порядка.
            Так как $d^2f$ - непрерывная функция на компакте, то у нее $\exists \min = C > 0$, т.е.\\ $\forall v \in S^{m-1} \quad d^2f(p)\langle v \rangle \geq C$.
            \begin{statement*}
                Тогда $\forall v \neq 0 \ d^2f(p)\langle v \rangle \geq C\cdot \left| v \right|^2$
                \begin{proof}
                    \[\forall v \neq 0 \quad d^2f(p)\langle v \rangle = d^f(p)\langle \left| v \right| \cdot \frac{v}{\left| v \right|} \rangle = 
                    \left| v \right|^2\cdot d^2f(p)\langle \frac{v}{\left| v \right|} \rangle \geq C\cdot \left| v \right|^2\]
                \end{proof}
            \end{statement*}

            Значит 
            \[
                f(x) = f(p) + df(p)\langle x -p \rangle + \frac{d^2f(p)\langle x - p \rangle}{2!} + \alpha(x)\left| x - p\right|^2, \ \alpha(x) \underset{x \to p }{\to} O(1)
            \]
            \[f(x) \geq f(p) + 0 + \frac{C}{2!}\cdot \left| x - p \right|^2 + \alpha(x)\left| x - p\right|^2\]
            Существует окрестность $U$ точки $p$, такая, что $\left| \alpha(x) \right| \leq \frac{C}{3} \ \forall x \in U$. Тогда для $\forall x \in U$:
            \[f(x) \geq f(p) + \frac{C}{2!}\left| x - p \right|^2 - \frac{C}{3}\left| x - p \right|^2 = f(p) + \frac{C}{6}\left| x - p\right| ^2\]
            То есть $f(x) - f(p) \geq \frac{C}{6}\left| x - p \right|^2 > 0 \implies$ в $U \ f(p) < f(x) \ \forall x \in U$. Пункт 1 доказан. 
        \end{proof}
        Пункт 2 доказывается аналогично пункту 1.
    \end{proof}
\end{theorem}

\begin{theorem}[Полиномиальное разложение композиции]
    Пусть $k \geq 0, f, g$ - функции, $A(x), B(y)$ - полиномы. Предположим, что $f$ и $A$ в точке $p$ имеют порядок касания $\geq k$, $g$ и $B$ в точке $p$ имеют порядок касания $\geq k$. То есть:
    \[f(x) - A(x) = \alpha(x) \underset{x \to p}{=} o(\left| x - p\right|^k), \ \alpha(p) = 0\] 
    \[g(y) - B(y) = \beta(y) \underset{y \to q}{=} o(\left| y - q \right|^k), \ \beta(q) = 0\]
    Тогда $g\circ f$ имеет с $B \circ A$ порядок касания $\geq k$. 
    \begin{proof}
        При $k = 0$:
        \[\alpha(x) = o(1) \implies f(x) - A(x) \underset{x \to p}{=} 0 \implies f(x) \underset{x \to p}{\to} f(p)\]
        Для функции $g$ аналогично, после чего применяем теорему о непрерывности композиции. Для $ k = 0$ доказано.
\newline
        Пусть $k \geq 0$. Тогда:
        \[\begin{cases}
            f(x) = \alpha(x) + A(x) \\
            g(y) = \beta(x) + B(x)
        \end{cases} \implies
        \begin{cases}
            \alpha(x) = o(\left| x - p \right|) \\
            \beta(x) = o(\left| y - q \right|)
        \end{cases} \implies 
        \begin{cases}
            \alpha \in D^1(p) \\
            \beta \in D^1(q)
        \end{cases} \implies 
        \begin{cases}
            f \in D^1(p) \\
            g \in D^1(q)
        \end{cases}\]
        \[g(f(x)) - B(A(x)) = 
        g(f(x)) - B(f(x)) + B(f(x)) - B(A(x)) = 
        \beta(f(x)) +B(f(x)) + B(A(x))\]
        Заметим, что $\beta(f(x)) = o(\left| f(x) - q\right|^k) \underset{x \to p }{=} o(\left| f(x)  - f(p) \right|^k)$. 
        При этом $f(x) - f(p) = O(\left| x - p \right|^1)$, так как $f \in D^1(p)$. А значит:
        \[\beta(f(x)) = o(O(\left| x - p \right|^k)) = o(\left| x - p \right|^k)\]
        \newline
        Пусть теперь $V$ - шар конечного радиуса с центром в $q$. Все частные производные многочлена 
        $B$ в шаре $V$ ограничены некоторой константой $C$. Тогда по лемме об оценке приращения:
        \[\forall y_1, y_2 \in V \quad \left| B(y_1) - B(y_2) \right| = O(\left| y_1 - y_2 \right|)\]
        При $x \to p \begin{cases}
            f(x) \to q \\
            A(x) \to q
        \end{cases}$ и поэтому $f(x) , B(x) \in V$.
        В таком случае \[
        B(f(x)) - B(A(x)) \underset{x \to p}{=} O(f(x) - A(x) ) = O(o(\left| x - p \right|^k))\]
    \end{proof}
\end{theorem}
    \section{Аналитические функции}

Пусть $w = f(z) = u(x, y) + iv(x,y)$, где $z = x + iy$ - функция, определенная в области $D \in \mathbb{C}$.

\begin{definition}
    Функция $f(z)$ называется дифференцируемой (моногенной), в точке $z \in D$, 
    если 
    \[\exists! \lim_{\Delta z \to 0} \frac{\Delta f}{\Delta z} = \lim_{\Delta z \to 0} \frac{f(z + \Delta z) - f(z)}{\Delta z}, \text{ где $z+ \Delta z \in D$}\]
    Этот предел называется производной функции $f$ и обозначается $f'(z)$
\end{definition}


\begin{note}
    Если этот предел существует, то он не зависит от того, как $\Delta z$ стремится к $0$.

    \begin{proof}
        \textbf{Шаг 1}\\
        Рассмотрим $\Delta z = \Delta x \to 0 \ (\Delta y = 0)$:
        \[f'(z) = \lim_{\Delta x \to 0} \left[\frac{u(x + \Delta x, y) - u(x, y)}{\Delta x} + i \frac{v(x + \Delta x, y) - v(x, y)}{\Delta x}\right] = \frac{\partial u}{\partial x} + i \frac{\partial v}{\partial x}\]
        \textbf{Шаг 2}\\
        Пусть теперь $\Delta z = i \Delta y \to 0 \ (\Delta x = 0)$:
        \[f'(z) = \lim_{\Delta y \to 0} \left[\frac{u(x, y+\Delta y) - u(x, y)}{i\Delta y} + i\frac{v(x, y+\Delta y) - v(x, y)}{i\Delta y}\right] = \frac{\partial v}{\partial y} - i \frac{\partial u}{\partial y}\]
        Для доказательства замечания нам необходимо, чтобы эти производные были равны, тогда:
        \[\begin{cases*}
            \frac{\partial u}{\partial x} = \frac{\partial v}{\partial y}\\
            \frac{\partial v}{\partial x} = -\frac{\partial u}{\partial y}
        \end{cases*} \text{ - условие Коши-Римана}\]
    \end{proof}
\end{note}

\begin{statement}
    Если $\exists f'(z)$ в точке $z \in D$, то выполнено условие Коши-Римана, но обратное утверждение не верно.
    \begin{proof}
        Слева направо ($\implies$) доказали в предыдущем замечании.
        Докажем справа налево ($\impliedby$).
        Построим контрпример:
        \[f(z) = \begin{cases*}
            e^{-\frac{1}{z^4}} \ z \neq 0 \\
            0 \ z = 0
        \end{cases*}\]
        Заметим, что если $f(0) = 0$, то $f'(0) = \lim_{z \to 0}\frac{f(z)}{z}$
        
        Пусть $x \to 0, y = 0$:
        \[\lim_{x \to 0}\frac{f(x)}{x} = \lim_{x \to 0}\frac{e^{-\frac{1}{x^4}}}{x} = u_x + iv_x = 0 \implies u_x = 0, v_x = 0\]
        Пусть $y \to 0, x = 0$:
        \[\lim_{y \to 0}\frac{f(iy)}{iy} = \lim_{y \to 0}\frac{e^{-\frac{1}{(iy)^4}}}{iy} = v_y + iu_y = 0 \implies v_y = 0, u_y = 0\]
        То есть $u_x = v_y = 0, v_x = -u_y = 0$. Значит условие Коши-Римана выполнено в $z = 0$,
        но с другой стороны, $f(z)$ не то, что не дифференцируема, она разрывна в $z = 0$:
        \par
        Пусть $z = (1+ i)x \to 0$.
        \[\lim_{x \to 0}\frac{f((1+i)x)}{(1+i)x} = \lim_{x \to 0}\frac{e^{-\frac{1}{(1+i)^4 x^4}}}{(1+i)x} = \lim_{x \to 0}\frac{e^{\frac{1}{4x^4}}}{(1+i)x} = \infty\]
        $\infty \neq 0 \implies$ функция $f(z)$ не непрерывна в 0 $\implies$ $\nexists f'(0)$.

        
    \end{proof}
\end{statement}

\begin{statement}
    Если функции $u(x,y), v(x,y)$ дифференцируемы в $z$ и выполнено условие Коши-Римана, то $\exists f'(z)$.
    \begin{proof}
        Так как $u, v$ - дифференцируемы в $z$, то
        \begin{equation}
            \begin{aligned}
            \Delta u &= \frac{\partial u}{\partial x}\Delta x + \frac{\partial u}{\partial y}\Delta y + o(\left| \Delta z \right|) \\
            \Delta v &= \frac{\partial v}{\partial x}\Delta x + \frac{\partial v}{\partial y}\Delta y + o(\left| \Delta z \right|)
            \end{aligned}
            \text{ где $\left|\Delta z \right| = \sqrt{\Delta x^2 + \Delta y^2}$}
            \label{deltauv}
        \end{equation} 
        Обозначим: 
        \begin{equation}
            \begin{aligned}
                \frac{\partial}{\partial z} = \frac{1}{2}\left(\frac{\partial}{\partial x} - i\frac{\partial}{\partial y} \right) \\
                \frac{\partial}{\partial \overline{z}} = \frac{1}{2}\left(\frac{\partial}{\partial x} + i\frac{\partial}{\partial y}\right)
            \end{aligned}
            \label{operators}
        \end{equation}
        Заметим, что 
        \begin{equation}
            \Delta x = \frac{1}{2}\left(\Delta z + \Delta \overline{z}\right), \ \Delta y = \frac{1}{2i}\left(\Delta z - \Delta \overline{z}\right)
            \label{rewrited}
        \end{equation}
        Тогда перепишем $\eqref{deltauv}$:
        \[\Delta f = \Delta u + i \Delta v = \left( \frac{\partial u}{\partial x} + i \frac{\partial v}{\partial x}\right)\Delta x + \left(\frac{\partial u}{\partial y} + i\frac{\partial v}{\partial y}\right)\Delta y + o(\Delta z) \overset{\text{по } \eqref{operators} \text{ и }\Delta x, \Delta y}{=}\frac{\partial f}{\partial z}\Delta z + \frac{\partial f}{\partial \overline{z}}\Delta \overline{z} + o(\Delta z)\]
        \begin{equation*}
        \begin{aligned}
            \frac{\partial f}{\partial z} &= \frac{1}{2}\left(\frac{\partial}{\partial x} - i\frac{\partial}{\partial y}\right)(u+iv) = \frac{1}{2}\left[\left(\frac{\partial u}{\partial x} + \frac{\partial v}{\partial y}\right) + i\left(\frac{\partial v}{\partial x} - \frac{\partial u}{\partial y}\right)\right] \\
            \frac{\partial f}{\partial \overline{z}} &= \frac{1}{2}\left(\frac{\partial}{\partial x} + i\frac{\partial}{\partial y}\right)(u+iv) = \frac{1}{2}\left[\left(\frac{\partial u}{\partial x} - \frac{\partial v}{\partial y}\right) + i\left(\frac{\partial v}{\partial x} + \frac{\partial u}{\partial y}\right)\right]
        \end{aligned}
    \end{equation*}
    Это выражение - формальная частная производная по $z$ и $\overline{z}$. Подставив эти    формулы в $\eqref{rewrited}$ получим
    \[\Delta f = \frac{\partial f}{\partial z}\Delta z + \frac{\partial f}{\partial \overline{z}}\Delta \overline{z} + o(\left| \Delta z \right|)\]
    Заметим, что $\frac{\partial f}{\partial \overline{z}} = 0 \iff$ выполнено условие Коши-Римана.Разделим предыдущее выражение на $\Delta z \neq 0$:
    \[\frac{\Delta f}{\Delta z} = \frac{\partial f}{\partial z} + \frac{o(\Delta z)}{\Delta z}\]
    Поскольку $\frac{o(\Delta z)}{\Delta z} \underset{\Delta z \to 0}{\to} 0$, то 
    \[\exists \lim_{\Delta z \to 0} \frac{\Delta f}{\Delta z} = \frac{\partial f}{\partial z} = f'(z)\]
\end{proof}
\end{statement}

\[\lim_{\Delta z \to 0} \frac{\Delta f}{\Delta z} = f'(z) \implies \frac{\Delta f}{\Delta z} = f'(z) + \eta, \ \lim_{\Delta z \to 0} \eta = 0\]
Значит $\Delta w = \Delta f = f'(z)\Delta z + \eta \cdot \Delta z$, где $f'(z)\Delta z$ - линейная часть приращения функции относительная $\Delta  z$, она же главная часть приращения, она же дифференциал функции.

Обозначим $dw = df(z) = f'(z)\Delta z$. В частности, если $f(z) = z$, то $df(z) = dz = \Delta z \implies df(z) = f'(z)dz$ или $f'(z) = \frac{df(z)}{dz}$.

\begin{definition}
    Функция $f(z)$ называется аналитичной в области $D$, если $\forall z \in D \ \exists f'(z)$.
\end{definition}

\begin{definition}
    Функция $f(z)$ называется аналитичной в точке $z_0 \in D$ ($D$ - область), если $f(z)$ аналитична в некоторой открытой окрестности точки $z_0$.
\end{definition}

\begin{theorem*}
    Сумма степенного ряда $S(z) = \sum_{k=1}^{\infty} C_k z^k$ аналитична в круге его сходимости $\left| z \right| < R$, 
    причем $S'(z) = \sum k C_k z^{k-1}$. 
    \begin{proof}
        Пусть ряд
        \begin{equation}
            \label{series}
            S_0(z) = \sum_{k = 1}^{\infty} kC_k z^{k-1}
        \end{equation}
        Заметим, что радиус сходимости ряда $S'_0(z)$ тоже равен $R$:
        \[\overline{\lim_{k \to \infty}} \sqrt[k]{\left| kC_k\right|} = \lim_{k \to \infty} \sqrt[k]{k} \cdot \overline{\lim} \sqrt[k]{\left| C_k\right|} = R \text{ - по теореме Коши-Адамара}\]
        \[S_0(0) = C_1; \ S_0(z) = \sum kC_k z^{k-1} = \frac{1}{z}\sum kC_k z^k\]
        Заметим, что
        \[k = (1+ (k^{\frac{1}{k}}-1))^k = 1 + k(k^{\frac{1}{k}} - 1) + \frac{k(k-1)}{2}(k^{\frac{1}{k}}-1)^2 + \hdots + (k^{\frac{1}{k}} - 1)^k \implies k > \frac{k(k-1)}{2}(k^{\frac{1}{k}}-1)^2\] \[\implies \left| k^2 - 1\right| < \sqrt{\frac{2}{\sqrt{k-1}}} < \varepsilon\]
        Пусть $z$ - произвольная точка круга $\left| z \right| < R$ и $\Delta z: \left| z + \Delta z \right| < R$.

        \[\left| \frac{S(z+\Delta z) - S(z)}{\Delta z} - S_0(z)\right| \overset{?}{<} \varepsilon\]
        \begin{equation}
            \label{module}
            \left| \frac{S(z + \Delta z) - S(z)}{\Delta z} -S_0(z) \right| \leq \left| \sum \left[ \frac{C_k(z + \Delta z)^k - C_k z^k}{\Delta z} - kC_k z^k \right] \right| = \left| \sum C_k \left[ \frac{(z + \Delta z)^k}{\Delta z}  \right] \right|
        \end{equation}

        Заметим, что первое слагаемое можно расписать как 
        
        \begin{multline}
            \label{binom}
            \frac{(z+\Delta z)^k}{\Delta z} = \frac{(z+\Delta z)^{k-1}(z+\Delta z)}{\Delta z} = \\ = (z+\Delta z)^{k-1} + \frac{z(z+\Delta z)^{k-1}}{\Delta z} \overset{\text{так же}}{=} (z+\Delta z)^{k-1} + z(z+\Delta z)^{k-2} + \hdots + z^{k-1} + \frac{z^k}{\Delta z}
        \end{multline}
    

        Тогда второе слагаемое из \eqref{module} сокращается с последним слагаемым \eqref{binom} и получаем
        \begin{multline*}
            \left| \sum C_k \left[ (z + \Delta z)^{k-1} + z(z + \Delta z)^{k-2} + \hdots + z^{k-1} - kz^{k-1} \right] \right| \overset{\text{дважды нер-во } \Delta\text{-ника}}{\leq} \\
            \leq \left| \sum_{k = 1}^{N} C_k\left[(z+\Delta z)^{k-1} + z(z +\Delta z)^{k-2} + \hdots + z^{k-1} - kz^{k-1}\right]\right| + \\
            \left| \sum_{k = N-1}^{\infty} C_k \left[ (z+ \Delta z)^{k-1} + z(z+ \Delta z)^{k-2} + \hdots + z^{k-1} \right]\right| + \left| \sum_{k=N-1}^{\infty} kC_k z^{k-1}\right|
        \end{multline*}
        Возьмем число $r: 0 < r < R$ и $\left| z + \Delta z \right| < r$. Из аболютной сходимости ряда \eqref{series} в круге $\left| z \right| < R \implies \forall \varepsilon > 0 \ \exists N = N(r, \varepsilon)$
        Введем новый ряд 
        \[\sum_{k=N+1}^{\infty} k\left| C_k \right| r^{k-1} < \frac{\varepsilon}{3}\]
        При таком $N$ второй и третий модули $< \frac{\varepsilon}{3}$ по критерию Коши абсолютной сходимости ряда \eqref{series}. А в первом модуле конечная сумма, которая стремится к 0 при $\Delta z \to 0$, то есть по определению предела, первый модуль тоже $< \frac{\varepsilon}{3}$
    \end{proof}
\end{theorem*}


\begin{note}
    Сумма степенного ряда $S'(z) = \sum kC_kz^{k-1}$ тоже аналитична в круге $\left| z \right| < R$, причем 
    \[S''(z) = \sum k(k-1)C_k z^{k-2}\]
    \[S^{(n)} = \sum k(k-1)\hdots(k-n+1)C_k z^{k-n} \overset{\text{при } z = 0}{\implies} C_n = \frac{S^{(n)}(z)}{n!}, \ n = 1, 2, \hdots\]
\end{note}
    \section{Многообразия в $\mathbb{R}^n$}

\begin{definition}
    Пусть $M$ - метрическое пространство (необязательно подмножество в $\mathbb{R}^n$). $M$ является $k$-мерным мноогобразием без края,
    если $\forall p \in M$ у точки $p \ \exists $ окрестность $U_p \open M$ гомеоморфная открытому шару в $\mathbb{R}^k$.
\end{definition}

\begin{definition}
    Гомеоморфизм - непрерывное отображение, обратное к которому тоже непрерывно.
\end{definition}

\begin{theorem}[Брауэра об инвариантности области (без док-ва).]
    \par Пусть $U \open \mathbb{R}^k$ и $f:U \to \mathbb{R}^k$ - непрерывна и инъективна. \newline
    Тогда $f(U) \open \mathbb{R}^k$ и $f^{-1}: f(U) \to U$ - тоже непрерывна.
\end{theorem}

\begin{definition}
    $M$ - $k$-мерное $C^r$-мноогобразие в $R^n$, если:
    \[\forall p \in M \ \exists U \in \mathcal{N}(p), \ U \open M \text{ такая, что } U \overset{C^r}{\cong} \text{ открытому шару в } \mathbb{R}^k\]
\end{definition}

\begin{statement*}
    Предыдущее утверждение эквивалентно требованиям:
    \begin{enumerate}
        \item \[ \exists U \in \mathcal{N}(p) \ U \open M \text{ такая, что } U \cong \text{ открытому подмножеству } \omega \in \mathbb{R}^k\]
        \item \[ \exists U \in \mathcal{N}(p) \ U \open M \text{ такая, что } U \cong \mathbb{R}^k\]
    \end{enumerate}
\end{statement*}

\begin{theorem*}[1 о регулярных решениях]
    Пусть $\Omega \open \mathbb{R}^n, \ f_1, \hdots, f_k: \Omega \to \mathbb{R}$ - $C^r$ гладкие функции.
    Непустое множество $M$ задано как 
    \[M = \{\overline{x} \in \Omega \mid 
    \begin{array}{c}
        f_1(x) = 0 \\
        \vdots \\
        f_k(x) = 0
    \end{array}\}, \text{ где $\overline{x}$ - регулярная точка}\]
    То есть $\mathrm{rank} \frac{\partial f_{1 \hdots k}}{\partial x_{1 \hdots n}} = k$
    Тогда $M$ - $n-k$-мерное $C^r$-мноогобразие без края.

    \begin{proof}
        Пусть $p \in M$. Можно считать, что последние $k$ столбцов линейно независимы, то есть $\left| \frac{\partial f_{1\hdots k}}{\partial x_{n-k+1 \hdots n}}\right| \neq 0$ - определитель матрицы Якоби.  
        По теореме о неявной функции существует такая окрестность $\tilde{U}$ точки $p$, $\tilde{U} \open \mathbb{R}^n$, такая, что $\tilde{U} \cap M$ - график некоторой функции:
        \[x_{n-k+1 \hdots n} = g(x_{1 \hdots n-k})\]
        Осталось показать, что график функции является мноогобразием $U \subset \mathbb{R}^k$.
        \begin{lemma*}
            График любого $C^r$ отображения $g$, определенного на открытом подмножестве - это мноогобразие, гомеоморфное $U$.
            \begin{proof}
                \[X \overset{g}{\to} Y\]
                График $\Gamma_g = \{(x, g(x)) \mid x \in X\}$
                \[U \overset{g}{\to} \Gamma_g \ - C^r \text{ отображение}\]
                \[x \overset{g}{\to} (x, g(x)) \overset{g^-1 - \text{проекция на X}}{\to} X\]
            \end{proof}
        \end{lemma*}
        По лемме теорема доказана.
    \end{proof}
\end{theorem*}

\begin{note}
    Если у градиента функции в точке $p$ хотя бы одна координата не равна $0$, то $p$ - регулярная.
\end{note}

\begin{statement*}
    $X \overset{C^r}{\cong} Y$ - если $X$ - $C^r$-многобразие, то $Y$ - тоже.
    \begin{proof}
        Пусть $\psi: X \to Y$ - $C^r$-изоморфизм. Пусть $q\in Y, p = \psi^{-1}(q)$, по условию существует открытая
        окрестность $V$ точки $p$, $C^r$-изоморфная $U \open \mathbb{R}^k$. 
        \[U \overset{\varphi}{\cong} V \overset{\psi}{\cong} \psi(V) \open Y\]
        $\psi(V)$ - прообраз $V$ под действием $\psi^{-1}, V = \psi^{-1}(\psi(V))$\\
        $\psi \circ \varphi$ - $C^r$-изоморфизм $U$ на окрестность точки $q$ в $X$.
    \end{proof}
\end{statement*}

\begin{lemma*}[о локальном вложении]
    Пусть $f = \begin{array}{c}
        f_1(x_1, \hdots, x_k) \\
        \vdots \\
        f_n(x_1, \hdots, x_k)
    \end{array}$, $U \open \mathbb{R}^k, f: U \to \mathbb{R}^n$.\\
    Если точка $\overline{x_0} \in U$ такая, что $\frac{\partial f}{\partial x}(x_0) = k$, то у точки $x_0$ существует окрестность $\tilde{U}$ такая, что $f(\tilde{U})$ - $k$-мерное многобразие.

    \begin{proof}
        \[\begin{pmatrix}
            \frac{\partial f_1}{\partial x_1} & \hdots & \frac{\partial f_1}{\partial x_k} \\
            \vdots & & \vdots\\
            \frac{\partial f_k}{\partial x_1} & \hdots & \frac{\partial f_k}{\partial x_k} \\
            \vdots & & \vdots\\
            \frac{\partial f_n}{\partial x_1} & \hdots & \frac{\partial f_n}{\partial x_k}
        \end{pmatrix}\]
        Переставим $f$ (если надо) и считаем первые $k$ строк невырожденными в $x_0$.
        Рассмотрим урезанное отображение $\psi = (\begin{array}{c} f_1 \\ \vdots \\ f_k \end{array}): U \to \mathbb{R}^k$\\
        По теореме о локальной обратимости существует окрестность $\tilde{U} \ni x_0$, такая, что 
        $\psi(\tilde{U}) \open \mathbb{R}^k$ и $\psi|_{\tilde{U}}: \tilde{U} \to \psi(\tilde{U})$ - $C^r$-изоморфизм.
        \[\psi(\tilde{U}) \overset{\psi^{-1}}{\to} \tilde{U}\]
        $\tilde{U}$ в свою очередь, отображается в $f(\tilde{U})$ данным отображением:
        \[\begin{array}{c}
            f_1(\tilde{x_1}, \hdots, \tilde{x_k}) \\
            \vdots \\
            f_k(\tilde{x_1}, \hdots, \tilde{x_k}) \\
            f_{\text{остальные индексы}}(\tilde{x_1}, \hdots, \tilde{x_k})
        \end{array}\]
        Причем $f_1, \hdots, f_k = (x_1, \hdots, x_k)$. 
        \[f_{\text{остальные}}(\tilde{x_1}, \hdots, \tilde{x_k}) = f_{\text{остальные}}\circ \psi(x_1, \hdots, x_k)\]
        $f(\tilde{U})= \{(\overline{x}, \psi(\overline{x})) \mid \overline{x} \in \tilde{U}\}$ - график отображения $\psi$, то есть по соответствующей лемме о графике это мноогобразие.
    \end{proof}
\end{lemma*}
\end{document}